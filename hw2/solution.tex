\documentclass[10pt]{article}

\usepackage[T1]{fontenc}
\usepackage[utf8]{inputenc}
%\usepackage{beton}
%\usepackage{ccfonts}
%\usepackage{concrete}
\usepackage{concmath}
\usepackage{eulervm}
\usepackage{amsmath,amsthm,amssymb}
\usepackage{mathtools}
\usepackage{multicol}
\usepackage{marginnote}
\usepackage{pgfplots}
\usepackage{float}
\usepackage{hyperref}
\usepackage{bbm}
\pgfplotsset{compat=1.5}

\usepackage{listings}
\usepackage{xcolor}
\definecolor{codegreen}{rgb}{0,0.6,0}
\definecolor{codegray}{rgb}{0.5,0.5,0.5}
\definecolor{codepurple}{rgb}{0.58,0,0.82}
\definecolor{backcolour}{rgb}{0.95,0.95,0.92}
\lstdefinestyle{mystyle}{
    backgroundcolor=\color{backcolour},   
    commentstyle=\color{codegreen},
    keywordstyle=\color{magenta},
    numberstyle=\tiny\color{codegray},
    stringstyle=\color{codepurple},
    basicstyle=\ttfamily\footnotesize,
    breakatwhitespace=false,         
    breaklines=true,                 
    captionpos=b,                    
    keepspaces=true,                 
    numbers=left,                    
    numbersep=5pt,                  
    showspaces=false,                
    showstringspaces=false,
    showtabs=false,                  
    tabsize=2
}

\lstset{style=mystyle}

\usepackage{mathtools}

\usepackage{wasysym}
\usepackage[margin=1.5in]{geometry} 
\usepackage{enumerate}
\index{\usepackage}\usepackage{multicol}

\newcommand{\N}{\mathbf{N}}
\newcommand{\Z}{\mathbb{Z}}

\newcommand{\R}{\mathbf{R}}
\newcommand{\C}{\mathbf{C}}
\newcommand{\Pbb}{\mathbb{P}}
\newcommand{\Fcal}{\mathcal{F}}
\newcommand{\Acal}{\mathcal{A}}
\newcommand{\Ecal}{\mathcal{E}}
\newcommand{\Ebb}{\mathbb{E}}
\newcommand{\Qbb}{\mathbb{Q}}


\renewcommand{\mathbf}{\mathbold}

\newenvironment{theorem}[2][Theorem]{\begin{trivlist}
  \item[\hskip \labelsep {\bfseries #1}\hskip \labelsep {\bfseries #2.}]}{\end{trivlist}}
\newenvironment{lemma}[2][Lemma]{\begin{trivlist}
  \item[\hskip \labelsep {\bfseries #1}\hskip \labelsep {\bfseries #2.}]}{\end{trivlist}}
\newenvironment{exercise}[2][Exercise]{\begin{trivlist}
  \item[\hskip \labelsep {\bfseries #1}\hskip \labelsep {\bfseries #2.}]}{\end{trivlist}}
\newenvironment{reflection}[2][Reflection]{\begin{trivlist}
  \item[\hskip \labelsep {\bfseries #1}\hskip \labelsep {\bfseries #2.}]}{\end{trivlist}}
\newenvironment{proposition}[2][Proposition]{\begin{trivlist}
  \item[\hskip \labelsep {\bfseries #1}\hskip \labelsep {\bfseries #2.}]}{\end{trivlist}}
\newenvironment{corollary}[2][Corollary]{\begin{trivlist}
  \item[\hskip \labelsep {\bfseries #1}\hskip \labelsep {\bfseries #2.}]}{\end{trivlist}}

\newenvironment{definition}[2][Definition]{\begin{trivlist}
  \item[\hskip \labelsep {\bfseries #1}\hskip \labelsep {\bfseries #2.}]}{\end{trivlist}}

\begin{document}
	
  \renewcommand{\qedsymbol}{\smiley}
	\title{Investments Class \\ Problem set 1}
	\author{Daniel Grosu, William Martin, Denis Steffen}
	
	\maketitle

  \begin{exercise}{1}(Utility Theory)

From the statement, we know that the random variables $Y$ and $\epsilon$ have the following distributions:
$$ Y \sim \mathcal{N}(\mu_Y,\sigma_Y^2), \quad \epsilon \sim \mathcal{N}(-\mu_\epsilon,\sigma_\epsilon^2)$$ but they are not independent and have correlation $\rho$. 
The covariance between the two is $Cov(Y,\epsilon) = \rho\sqrt{\sigma_Y^2}\sqrt{\sigma_\epsilon^2} = \rho\sigma_\epsilon\sigma_\epsilon$. 

We have to consider the random vector $\mathbf{Z} = (Y,\epsilon)^\top$ which has multivariate gaussian distribution because every linear combination of the components $\alpha Y + \beta \epsilon$ is normally distributed. Thus:
$$ \mathbf{Z} \sim \mathcal{N}_2 \left( {\mu_Y \choose -\mu_\epsilon},\begin{bmatrix}
  \sigma^2_Y&\rho\sigma_Y\sigma_\epsilon\\
  \rho\sigma_Y\sigma_\epsilon&\sigma_\epsilon^2\\
  \end{bmatrix} \right) $$

  First, we can calculate the relative risk aversion and the absolute risk aversion:
\begin{equation*}
  \text{ARA}(W) = -\frac{u''(W)}{u'(W)} = \frac{a^2e^{-aW}}{ae^{-aW}} = a, \quad \text{RRA}(W) = -\frac{Wu''(W)}{u'(W)} = -aW
\end{equation*} since $u'(W) = ae^{-aW}$ and $u''(W) = -a^2e^{-aW}$.

Then, we want to compute the insurance premium $\pi$ to insure the risk of $\epsilon$ such that: $$ \Ebb[u(Y+\epsilon)] = \Ebb[u(Y-\pi)] $$
We can first compute the right hand side expectation using the moments generating function of a multivariate gaussian random variable $\mathcal{N}_n(\mathbf{\mu},\mathbf{\Sigma}) $:
$$ MGF_\mathbf{X}(t) = \Ebb[e^{\mathbf{t}^\top\mathbf{Z}}] = \exp(\mathbf{t}^\top\mathbf{\mu} + \frac{1}{2}\mathbf{t}^\top\mathbf{\Sigma}\mathbf{t})$$
Thus, 
\begin{align*}
  \Ebb[e^{-a(Y+\epsilon)}] = \Ebb[e^{-a\mathbbm{1}^\top\mathbf{Z}}] &= \exp(-a\mathbbm{1}^\top\mu + \frac{1}{2}a^2\mathbbm{1}^\top\mathbf{\Sigma}\mathbbm{1}) \\
  &= \exp(-a(\mu_Y - \mu_\epsilon)+\frac{1}{2}a^2(\sigma_Y^2 + 2\rho\sigma_Y\sigma_\epsilon + \sigma_\epsilon^2))
\end{align*}

On the other side, we can see that:
\begin{align*}
  \Ebb[e^{-a(Y-\pi)}] &= e^{a\pi}\Ebb[e^{-aY}] \\
  &= e^{a\pi}e^{-a\mu_Y + \frac{1}{2}a^2\sigma_Y^2}
\end{align*}
 
Combining the two, we get:
\begin{align*}
  &e^{-a(\mu_Y-\mu_\epsilon) +\frac{1}{2}a^2(\sigma_Y^2 + 2\rho\sigma_Y\sigma_\epsilon + \sigma_\epsilon^2)} = e^{a\pi}e^{-a\mu_Y + \frac{1}{2}a^2\sigma_Y^2}\\
  &\iff e^{a\pi} = e^{a\mu_\epsilon +\frac{1}{2}a^2(2\rho\sigma_Y\sigma_\epsilon + \sigma_\epsilon^2)}\\
  &\iff \pi = \mu_\epsilon + \frac{1}{2}a(2\rho\sigma_Y\sigma_\epsilon + \sigma_\epsilon^2)
\end{align*}
Finally, the maximum insurance premium we are willing to pay to insure the risk of $\epsilon$ is $\mu_\epsilon + \frac{1}{2}a(2\rho\sigma_Y\sigma_\epsilon + \sigma_\epsilon^2)$.

Furthermore, we can study how the premium evolves according to the five parameters of the random variable distributions. 
We differentiate $\pi$ with respect to these variables:
\begin{itemize}
  \item $\frac{\partial\pi}{\partial\mu_\epsilon} = 1$
  \item $\frac{\partial\pi}{\partial a} = \rho\sigma_Y\sigma_\epsilon + \sigma_\epsilon^2$
  \item $\frac{\partial\pi}{\partial\rho} = a\sigma_Y\sigma_\epsilon$
  \item $\frac{\partial\pi}{\partial\sigma_\epsilon} = a\rho\sigma_Y + \sigma_\epsilon$
  \item $\frac{\partial\pi}{\partial\sigma_Y} = a\rho\sigma_\epsilon$
\end{itemize}
  \end{exercise}
\end{document}

