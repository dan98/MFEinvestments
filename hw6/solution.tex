\documentclass[10pt]{article}

\usepackage[T1]{fontenc}
\usepackage[utf8]{inputenc}
%\usepackage{beton}
%\usepackage{ccfonts}
%\usepackage{concrete}
\usepackage{concmath}
\usepackage{eulervm}
\usepackage{amsmath,amsthm,amssymb}
\usepackage{mathtools}
\usepackage{multicol}
\usepackage{marginnote}
\usepackage{pgfplots}
\usepackage{float}
\usepackage{hyperref}
\usepackage{bbm}
\usepackage{booktabs}
\pgfplotsset{compat=1.5}

\usepackage{listings}
\usepackage{xcolor}
\definecolor{codegreen}{rgb}{0,0.6,0}
\definecolor{codegray}{rgb}{0.5,0.5,0.5}
\definecolor{codepurple}{rgb}{0.58,0,0.82}
\definecolor{backcolour}{rgb}{0.95,0.95,0.92}
\lstdefinestyle{mystyle}{
    backgroundcolor=\color{backcolour},   
    commentstyle=\color{codegreen},
    keywordstyle=\color{magenta},
    numberstyle=\tiny\color{codegray},
    stringstyle=\color{codepurple},
    basicstyle=\ttfamily\footnotesize,
    breakatwhitespace=false,         
    breaklines=true,                 
    captionpos=b,                    
    keepspaces=true,                 
    numbers=left,                    
    numbersep=5pt,                  
    showspaces=false,                
    showstringspaces=false,
    showtabs=false,                  
    tabsize=2
}

\lstset{language=Python, style=mystyle}

\usepackage{mathtools}

\usepackage{wasysym}
\usepackage[margin=1.5in]{geometry} 
\usepackage{enumerate}
\index{\usepackage}\usepackage{multicol}

\newcommand{\N}{\mathbf{N}}
\newcommand{\Z}{\mathbb{Z}}

\newcommand{\R}{\mathbf{R}}
\newcommand{\C}{\mathbf{C}}
\newcommand{\Pbb}{\mathbb{P}}
\newcommand{\Fcal}{\mathcal{F}}
\newcommand{\Lcal}{\mathcal{L}}
\newcommand{\Acal}{\mathcal{A}}
\newcommand{\Ecal}{\mathcal{E}}
\newcommand{\Ebb}{\mathbb{E}}
\newcommand{\Qbb}{\mathbb{Q}}


\renewcommand{\mathbf}{\mathbold}

\newenvironment{theorem}[2][Theorem]{\begin{trivlist}
  \item[\hskip \labelsep {\bfseries #1}\hskip \labelsep {\bfseries #2.}]}{\end{trivlist}}
\newenvironment{lemma}[2][Lemma]{\begin{trivlist}
  \item[\hskip \labelsep {\bfseries #1}\hskip \labelsep {\bfseries #2.}]}{\end{trivlist}}
\newenvironment{exercise}[2][Exercise]{\begin{trivlist}
  \item[\hskip \labelsep {\bfseries #1}\hskip \labelsep {\bfseries #2.}]}{\end{trivlist}}
\newenvironment{reflection}[2][Reflection]{\begin{trivlist}
  \item[\hskip \labelsep {\bfseries #1}\hskip \labelsep {\bfseries #2.}]}{\end{trivlist}}
\newenvironment{proposition}[2][Proposition]{\begin{trivlist}
  \item[\hskip \labelsep {\bfseries #1}\hskip \labelsep {\bfseries #2.}]}{\end{trivlist}}
\newenvironment{corollary}[2][Corollary]{\begin{trivlist}
  \item[\hskip \labelsep {\bfseries #1}\hskip \labelsep {\bfseries #2.}]}{\end{trivlist}}

\newenvironment{definition}[2][Definition]{\begin{trivlist}
  \item[\hskip \labelsep {\bfseries #1}\hskip \labelsep {\bfseries #2.}]}{\end{trivlist}}

\begin{document}
	
  \renewcommand{\qedsymbol}{\smiley}
	\title{Investments Class \\ Problem set 6}
	\author{Daniel Grosu, William Martin, Denis Steffen}
		
\maketitle

\begin{exercise}{1}
  \begin{itemize}
    \item We can first compute the expected stock return using the formula of the model, and we obtain by linearity of the expectation: 
     $$ \Ebb[R_i] = \alpha_i + \sum_{k=1}^K B_{ik}\Ebb[F_k] + \Ebb[\epsilon_i] = \alpha_i + \sum_{k=1}^K B_{ik}m_k $$
     The Arbitrage Pricing Theory requires that the expected stock return has the following form: $$ \Ebb[R_i] = R_0 + \sum_{k=1}^K B_{ik}m_k^e$$ where $m_k^e$ are the expected excess returns ($m_k^e = m_k - R_0$). 
     So, we obtain the condition for $\alpha_i$: 
     $$ \alpha_i = R_0 - R_0\sum_{k=1}^K B_{ik} = R_0(1-\sum_{k=1}^KB_{ik})$$
    \item If the market portfolio is spanned by the factors, we have $$ R_M = \sum_{k=1}^Kw_kF_k$$ And the CPAM holds if $$ \Ebb[R_i] = R_0 + \beta_i(\Ebb[R_M]-R_0) = R_0 + \frac{Cov(R_i,R_M)}{\sigma_M^2}(\sum_{k=1}^Kw_km_k-R_0)$$
    $$ = R_0 + \left(\frac{1}{\sigma_M^2}\sum_{k=1}^KB_{ik}Cov(F_k,R_M)\right)(\sum_{k=1}^Kw_km_k-R_0)$$ because the $(F_k)_{1\leq k\leq K}$ random variables are independent.
    \\
    So, we can rewrite the expectation using $\beta_{F_k} = \frac{Cov(F_k,R_M)}{\sigma_M^2}$ as: 
    \begin{align*}
      \Ebb[R_i] &= R_0 + \left(\sum_{k=1}^K B_{ik}\beta_{F_k}\right)\left(\sum_{k=1}^Kw_km_k-R_0\right) \\
      &= R_0 + \sum_{k=1}^K B_{ik}(\beta_{F_k}(\Ebb[R_M]-R_0)) \\
      &= R_0 + \sum_{k=1}^K B_{ik}\lambda_k
    \end{align*} where $ \lambda_k = \beta_{F_k}(\Ebb[R_M]-R_0) = \Ebb[F_k] - R_0$, by the CPAM relation for $F_k$.

    Thus, we know that the APT must hold: 
    $$ \Ebb[R_i] = R_0 + \sum_{k=1}^K B_{ik}(\Ebb[F_k] - R_0) = R_0 + \sum_{k=1}^K B_{ik}(m_k^e)$$ and this is true if and only if the CPAM holds.
    We need to have: $$ \beta_{F_k}(\sum_{l=1}^K w_lm_l - R_0) = m_k - R_0$$ and so $$ \sum_{l=1}^K w_lm_l = R_0 + \frac{m_k-R_0}{\beta_{F_k}}\quad \text{(risk premia)}$$
    and $$ \sigma_M^2 = \sum_{k=1}^K w_k^2(\sigma_k^2 + m_k^2)-(\sum_{k=1}^Kw_km_k)^2 \quad \text{(volatility)}$$
    \item In this case we have that: 
    $$ \Ebb[R_i] = R_0 + \sum_{k=1}^K B_{ik}\lambda_k$$ and so for $N$ assets in portfolio $p$:
    $$ \Ebb[R_p] = \frac{1}{N}\sum_{i=1}^N\Ebb[R_i] = \frac{1}{N}\sum_{i=1}^N(R_0 + \sum_{k=1}^K B_{ik}\lambda_k)$$, using that $B_{ik}\geq b \forall (i,k)$
    $$ \Ebb[R_p] \geq \frac{1}{N}\left(NR_0 + N\sum_{k=1}^K b\lambda_k\right) = R_0 + b\sum_{k=1}^K\lambda_k > 0 $$ because $b> 0, \lambda_k >0 \text{ and } R_0 \geq 0$.

    Hence, we see that the returns do not converge to zero when the number of assets $N$ tends to infinity. This contradicts the Arbitrage Pricing Theory on fully diversified portfolios that requires $ \frac{1}{N}\sum_{i=1}^Nw_i\nu_i \rightarrow 0$ when $N \rightarrow \infty$.  

    We can see that the volatility of each asset is $V(R_i) = \sigma_{\epsilon}$ because $\sum_{k=1}^KB_{ik}\lambda_k$ is deterministic and known. 
    So for the portfolio: 
    $$ V(R_p) = \frac{1}{N^2}\sum_{i=1}^N\sigma_{\epsilon} = \frac{1}{N}\sigma_\epsilon \rightarrow 0 \quad \text{when } N \rightarrow \infty$$
    So the asymptotic arbitrage portfolio is given by $w = (\frac{1}{N},\dots,\frac{1}{N})$ 
    This portfolio has stricly positive expected profits (as $\Ebb[R_p] > 0$) and no risk (as $V(R_p)\rightarrow 0$).
  \end{itemize}
	
\end{exercise}
  
\end{document}
