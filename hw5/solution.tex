\documentclass[10pt]{article}

\usepackage[T1]{fontenc}
\usepackage[utf8]{inputenc}
%\usepackage{beton}
%\usepackage{ccfonts}
%\usepackage{concrete}
\usepackage{concmath}
\usepackage{eulervm}
\usepackage{amsmath,amsthm,amssymb}
\usepackage{mathtools}
\usepackage{multicol}
\usepackage{marginnote}
\usepackage{pgfplots}
\usepackage{float}
\usepackage{hyperref}
\usepackage{bbm}
\usepackage{booktabs}
\pgfplotsset{compat=1.5}

\usepackage{listings}
\usepackage{xcolor}
\definecolor{codegreen}{rgb}{0,0.6,0}
\definecolor{codegray}{rgb}{0.5,0.5,0.5}
\definecolor{codepurple}{rgb}{0.58,0,0.82}
\definecolor{backcolour}{rgb}{0.95,0.95,0.92}
\lstdefinestyle{mystyle}{
    backgroundcolor=\color{backcolour},   
    commentstyle=\color{codegreen},
    keywordstyle=\color{magenta},
    numberstyle=\tiny\color{codegray},
    stringstyle=\color{codepurple},
    basicstyle=\ttfamily\footnotesize,
    breakatwhitespace=false,         
    breaklines=true,                 
    captionpos=b,                    
    keepspaces=true,                 
    numbers=left,                    
    numbersep=5pt,                  
    showspaces=false,                
    showstringspaces=false,
    showtabs=false,                  
    tabsize=2
}

\lstset{language=Python, style=mystyle}

\usepackage{mathtools}

\usepackage{wasysym}
\usepackage[margin=1.5in]{geometry} 
\usepackage{enumerate}
\index{\usepackage}\usepackage{multicol}

\newcommand{\N}{\mathbf{N}}
\newcommand{\Z}{\mathbb{Z}}

\newcommand{\R}{\mathbf{R}}
\newcommand{\C}{\mathbf{C}}
\newcommand{\Pbb}{\mathbb{P}}
\newcommand{\Fcal}{\mathcal{F}}
\newcommand{\Lcal}{\mathcal{L}}
\newcommand{\Acal}{\mathcal{A}}
\newcommand{\Ecal}{\mathcal{E}}
\newcommand{\Ebb}{\mathbb{E}}
\newcommand{\Qbb}{\mathbb{Q}}


\renewcommand{\mathbf}{\mathbold}

\newenvironment{theorem}[2][Theorem]{\begin{trivlist}
  \item[\hskip \labelsep {\bfseries #1}\hskip \labelsep {\bfseries #2.}]}{\end{trivlist}}
\newenvironment{lemma}[2][Lemma]{\begin{trivlist}
  \item[\hskip \labelsep {\bfseries #1}\hskip \labelsep {\bfseries #2.}]}{\end{trivlist}}
\newenvironment{exercise}[2][Exercise]{\begin{trivlist}
  \item[\hskip \labelsep {\bfseries #1}\hskip \labelsep {\bfseries #2.}]}{\end{trivlist}}
\newenvironment{reflection}[2][Reflection]{\begin{trivlist}
  \item[\hskip \labelsep {\bfseries #1}\hskip \labelsep {\bfseries #2.}]}{\end{trivlist}}
\newenvironment{proposition}[2][Proposition]{\begin{trivlist}
  \item[\hskip \labelsep {\bfseries #1}\hskip \labelsep {\bfseries #2.}]}{\end{trivlist}}
\newenvironment{corollary}[2][Corollary]{\begin{trivlist}
  \item[\hskip \labelsep {\bfseries #1}\hskip \labelsep {\bfseries #2.}]}{\end{trivlist}}

\newenvironment{definition}[2][Definition]{\begin{trivlist}
  \item[\hskip \labelsep {\bfseries #1}\hskip \labelsep {\bfseries #2.}]}{\end{trivlist}}

\begin{document}
	
  \renewcommand{\qedsymbol}{\smiley}
	\title{Investments Class \\ Problem set 5}
	\author{Daniel Grosu, William Martin, Denis Stiffen}
		
\maketitle

\begin{exercise}{1}
  \begin{itemize}
    \item The tangency portfolio is the optimal weights solution for the variance minimization problem:
    $$ \min_w{\frac{1}{2}w'\Sigma w} \quad \text{ with the constraints } (\mu-R_0\mathbbm{1})'w = \mu_p-R_0 \text{ and } w'\mathbbm{1} = 1$$ which can be solved by using a Lagrangian function: $L(w,\lambda) =  \frac{1}{2}w'\Sigma w - \lambda((\mu-R_0\mathbbm{1})'w - \mu_p-R_0)$. 
    \\
    We derive with respect to the two parameters:
    \begin{align*}
      &\frac{\partial L}{\partial w} = \Sigma w -\lambda (\mu-R_0\mathbbm{1}) = 0 \\
      &\frac{\partial L}{\partial \lambda} = (\mu-R_0\mathbbm{1})'w - \mu_p-R_0 = 0 
    \end{align*}
    And, thus: $w = \lambda\Sigma^{-1}(\mu-R_0\mathbbm{1})$. $\lambda$ can be calculated using the second constraint (the sum of weights equals $1$, only risky-assets in the portfolio). So, we obtain:
    $$ \lambda = (\mathbbm{1}'\Sigma^{-1}(\mu-R_0\mathbbm{1}))^{-1}$$
    
    Using the course notations, with $A = \mathbbm{1}'\Sigma^{-1}\mu$ and $B=\mathbbm{1}'\Sigma^{-1}\mathbbm{1}$, we end up with:
    $$ w_t = \frac{1}{B-AR_0}\Sigma^{-1}(\mu-R_0\mathbbm{1})$$
    
    In our case, we have $3$ risky assets. The covariance matrix equals:
    $$ \Sigma = \left[ {\begin{array}{ccc}
      0.0225  &  0.0075  &  0.0090\\
      0.0075  &  0.0625  &  0.0150\\
      0.0090  &  0.0150  &  0.0900
    \end{array} } \right]$$ since $Cov(i,j) = Corr(i,j)\sigma_i\sigma_j$. Because $A = 54.4444$ and $B = 5.4778$, so the tangency portfolio has the following weights: $$ w_t = (0.4407, 0.2886, 0.2707)$$ 
    The tangency mean and standard deviation are $\mu_t = w_t'\mu = 11.22\%$ and $\sigma_t = \sqrt{w_t'\Sigma w_t} = 15.02\%$. Its Sharpe Ratio is $SR_t = \frac{\mu_t - R_0}{\sigma_t} = 0.4140$. 
    \item To find the zero beta portfolio $w_z$, the portfolio has to satisfy $w_z'\Sigma w_t = 0$ and is of the form $w = \lambda\Sigma^{-1}\mathbbm{1}+ \gamma\Sigma^{-1}\mu$ since it is also a mean-variance portfolio. Using the previous results, we can rewrite it as:
    \begin{align*}
      0 &= w_z'\Sigma(\frac{1}{B-AR_0}\Sigma^{-1}(\mu-R_0\mathbbm{1})) \\
      &= \frac{1}{B-AR_0}w_z'(\mu-R_0\mathbbm{1}) \\
      &= \frac{1}{B-AR_0}(w_z'\mu - R_0w_z'\mathbbm{1}) \\
      &= \frac{1}{B-AR_0}(\mu_z - R_0)
    \end{align*} because the portfolio is invested in risky assets only, so $w_z'\mathbbm{1} = 1$. And we require $\mu_z = R_0 = 5\%$. 
    \\
    We know $\mu_z$, so we can now compute the parameters $\lambda = \frac{C-\mu_zB}{AC-B^2}$ and $\gamma = \frac{\mu_zA-B}{AC-B^2}$ with $C = \mu'\Sigma^{-1}\mu$. 
    We obtain $C = 0.5830$ and so $\lambda= 0.1779$ and $\gamma = -1.5858$. 
    Finally, the zero beta portfolio is: 
    $$ w_z = (1.9099, -0.2748, -0.6351)$$
    And the corresponding mean, standard deviation and Sharpe ratio: 
    $$ \mu_z = 5\%, \quad \sigma_z = 31.41\%, \quad SR_z = 0$$
    \item The investor can invest in the tangency and the zero beta portfolio, so her portfolio can be written as $w_P = x_tw_t + x_zw_z$, with coefficients $x_t,x_z$ as described in the problem set. The investor will invest in both risky-assets and in the risk free rate because $\mathbbm{1}'w_P = \mathbbm{1}'(x_tw_t+x_zw_z) = x_t\mathbbm{1}'w_t + x_z\mathbbm{1}'w_z = x_t + w_t \leq m$ since the tangency and zero beta portfolios are invested in risky assets exclusively. 
    \\ However, $m =1.2\geq 1$, so she can invest in both risky portfolios and in the risk free rate.  

    As given, we have to maximize the objective function: $$ \max_w \Ebb[R_P] - \frac{a}{2}V[R_P]$$ subject to two constraints: $R_P = R_0 + x_t(R_t - R_0)  + x_z(R_z-R_0), \quad \mathbbm{1}'w_p \leq m$. 
    \\
    The Lagrangian for this problem is: \begin{align*}
      L(x_t,x_z) &= \Ebb[R_P] - \frac{a}{2}V[R_P] - \lambda(x_t+x_z -m)\\
      &= R_0 +x_t(\mu_t-R_0) + x_z(\mu_z-R_0) - \frac{a}{2}(x_t^2\sigma_t^2+x_z^2\sigma_z^2) - \lambda(x_t+x_z-m)
    \end{align*} and we differentiate it with respect to the two coefficients $x_t$ and $x_z$: 
    \begin{align*}
      &\frac{\partial L}{\partial x_t} = \mu_t-R_0 -ax_t\sigma_t^2-\lambda = 0\\
      &\frac{\partial L}{\partial x_z} = \mu_z-R_0 -ax_z\sigma_z^2-\lambda = 0
    \end{align*}
    But we require the second constraint to be an inequality, so the first-order condition for second constraint becomes $ \lambda\geq 0 \text{ and } \lambda(x_t+x_z-m) = 0$ (KKT conditions). 

    This is solved by: 
    \begin{align*}
      x_t &= \frac{\mu_t -R_0 -\lambda}{a\sigma_t^2}\\
      x_z &= \frac{\mu_z -R_0 -\lambda}{a\sigma_z^2}
    \end{align*} and using the Sharpe ratio formulas, and the fact that $SR_z = 0$:
    \begin{align*}
      x_t &= \frac{1}{a}(SR_t/\sigma_t - \lambda/\sigma_t^2)\\
      x_z &= \frac{1}{a}(-\lambda/\sigma_z^2)
    \end{align*}
    For $\lambda$, we have two cases: if $x_t+x_z < m$, then the KKT conditions imply that $\lambda = 0$. In this case, the investor does not invest in the zero beta portfolio ($x_z = 0$ and $x_t = SR_t/{a\sigma_t}$). The second case is $x_t+x_z = m$ and using the expressions for $x_t$ and $x_z$ we get:
    $$ \frac{1}{a}(SR_t\sigma_t - \lambda/\sigma_t^2) + \frac{1}{a}(-\lambda/\sigma_z^2) = m$$
    and thus $$ \lambda = (SR_t/\sigma_t -am)\frac{\sigma_t^2\sigma_z^2}{\sigma_t^2+\sigma_z^2}$$

    So the portfolio becomes, in the first case: 
    \begin{align*}
      w_p = x_tw_t = \frac{1}{a}\frac{SR_t}{\sigma_t} w_t &= \frac{1}{a}\frac{\mu_t-R_0}{\sigma_t^2}\frac{1}{B-AR_0}\Sigma^{-1}(\mu-R_0\mathbbm{1}) \quad \text{in the first case} \\
      &= \frac{1}{a}\Sigma^{-1}(\mu-R_0\mathbbm{1})
    \end{align*} since $$SR_t/\sigma_t = \frac{\frac{C-BR_0}{B-AR_0}- R_0}{\frac{C-2R_0B+R_0^2A}{(B-AR_0)^2}} = \frac{\frac{C-BR_0-R_0B+AR_0^2}{B-AR_0}}{\frac{C-2R_0B+R_0^2A}{(B-AR_0)^2}} = B-AR_0.$$

    In the second case, we obtain: 
  \begin{align*} 
  &w_p = x_tw_t + x_zw_z \\ 
  &= \frac{1}{a}((B-AR_0)(1-\frac{\sigma_z^2}{\sigma_t^2+\sigma_z^2})+ am\frac{\sigma_z^2}{\sigma_t^2+\sigma_z^2})w_t + \frac{1}{a}(am\frac{\sigma_t^2}{\sigma_t^2+\sigma_z^2}-(B-AR_0)\frac{\sigma_t^2}{\sigma_t^2+\sigma_z^2})w_z \\
  &= \frac{1}{a}(1 - \frac{\sigma_z^2}{\sigma_t^2+\sigma_z^2})\Sigma^{-1}(\mu-R_0\mathbbm{1}) + m\frac{\sigma_z^2}{\sigma_t^2+\sigma_z^2}\frac{1}{B-AR_0}\Sigma^{-1}(\mu-R_0\mathbbm{1}) + x_zw_z
  \end{align*} and cannot be expressed in a simpler form. 

  Now we give the mean, standard deviation and Sharpe ratios in the two cases:
  \\
  First, commonly known: 
  $$ \mu_p = \frac{1-B/a}{A}B + \frac{C}{a}$$
  $$ \sigma_p = \sqrt{\frac{1-B/a}{A} + \frac{\mu_p}{a}}$$
  $$ SR_p = \frac{\mu_p-R_0}{\sigma_p} $$
  
  And in the second case:
  $$ \mu_p = x_t\mu_t + x_z\mu_z = x_t(\mu_t -R_0) + mR_0$$
  $$ \sigma_p = \sqrt{x_t^2\sigma_t^2 + x_z^2\sigma_z^2}$$
  $$ SR_p = \frac{\mu_p -R_0 }{\sqrt{x_t^2\sigma_t^2 + x_z^2\sigma_z^2}}$$

\item The two previous cases imply two cases for the risk aversion, meaning there exists $a^\star$ that separates $a$ in each case. 
\\
In the first case, we have: $\frac{SR_t}{a\sigma_t} < m$ and thus $a > \frac{SR_t}{m\sigma_t}$. So we can set $a^\star :=  \frac{SR_t}{m\sigma_t}$. 
  
  
  \end{itemize}

\end{exercise}
  
\end{document}
