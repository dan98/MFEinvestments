\documentclass[10pt]{article}

\usepackage[T1]{fontenc}
\usepackage[utf8]{inputenc}
%\usepackage{beton}
%\usepackage{ccfonts}
%\usepackage{concrete}
\usepackage{concmath}
\usepackage{eulervm}
\usepackage{amsmath,amsthm,amssymb}
\usepackage{mathtools}
\usepackage{multicol}
\usepackage{marginnote}
\usepackage{pgfplots}
\usepackage{float}
\usepackage{hyperref}
\usepackage{bbm}
\usepackage{booktabs}
\pgfplotsset{compat=1.5}

\usepackage{listings}
\usepackage{xcolor}
\definecolor{codegreen}{rgb}{0,0.6,0}
\definecolor{codegray}{rgb}{0.5,0.5,0.5}
\definecolor{codepurple}{rgb}{0.58,0,0.82}
\definecolor{backcolour}{rgb}{0.95,0.95,0.92}
\lstdefinestyle{mystyle}{
    backgroundcolor=\color{backcolour},   
    commentstyle=\color{codegreen},
    keywordstyle=\color{magenta},
    numberstyle=\tiny\color{codegray},
    stringstyle=\color{codepurple},
    basicstyle=\ttfamily\footnotesize,
    breakatwhitespace=false,         
    breaklines=true,                 
    captionpos=b,                    
    keepspaces=true,                 
    numbers=left,                    
    numbersep=5pt,                  
    showspaces=false,                
    showstringspaces=false,
    showtabs=false,                  
    tabsize=2
}

\lstset{language=Python, style=mystyle}

\usepackage{mathtools}

\usepackage{wasysym}
\usepackage[margin=1.5in]{geometry} 
\usepackage{enumerate}
\index{\usepackage}\usepackage{multicol}

\newcommand{\N}{\mathbf{N}}
\newcommand{\Z}{\mathbb{Z}}

\newcommand{\R}{\mathbf{R}}
\newcommand{\C}{\mathbf{C}}
\newcommand{\Pbb}{\mathbb{P}}
\newcommand{\Fcal}{\mathcal{F}}
\newcommand{\Lcal}{\mathcal{L}}
\newcommand{\Acal}{\mathcal{A}}
\newcommand{\Ecal}{\mathcal{E}}
\newcommand{\Ebb}{\mathbb{E}}
\newcommand{\Qbb}{\mathbb{Q}}


\renewcommand{\mathbf}{\mathbold}

\newenvironment{theorem}[2][Theorem]{\begin{trivlist}
  \item[\hskip \labelsep {\bfseries #1}\hskip \labelsep {\bfseries #2.}]}{\end{trivlist}}
\newenvironment{lemma}[2][Lemma]{\begin{trivlist}
  \item[\hskip \labelsep {\bfseries #1}\hskip \labelsep {\bfseries #2.}]}{\end{trivlist}}
\newenvironment{exercise}[2][Exercise]{\begin{trivlist}
  \item[\hskip \labelsep {\bfseries #1}\hskip \labelsep {\bfseries #2.}]}{\end{trivlist}}
\newenvironment{reflection}[2][Reflection]{\begin{trivlist}
  \item[\hskip \labelsep {\bfseries #1}\hskip \labelsep {\bfseries #2.}]}{\end{trivlist}}
\newenvironment{proposition}[2][Proposition]{\begin{trivlist}
  \item[\hskip \labelsep {\bfseries #1}\hskip \labelsep {\bfseries #2.}]}{\end{trivlist}}
\newenvironment{corollary}[2][Corollary]{\begin{trivlist}
  \item[\hskip \labelsep {\bfseries #1}\hskip \labelsep {\bfseries #2.}]}{\end{trivlist}}

\newenvironment{definition}[2][Definition]{\begin{trivlist}
  \item[\hskip \labelsep {\bfseries #1}\hskip \labelsep {\bfseries #2.}]}{\end{trivlist}}

\begin{document}
	
  \renewcommand{\qedsymbol}{\smiley}
	\title{Investments Class \\ Problem set 5}
	\author{Daniel Grosu, William Martin, Denis Stiffen}
		
\maketitle

\begin{exercise}{1}
  \begin{itemize}
    \item The tangency portfolio is the optimal weights solution for the variance minimization problem:
    $$ \min_w{\frac{1}{2}w'\Sigma w} \quad \text{ with the constraints } (\mu-R_0\mathbbm{1})'w = \mu_p-R_0 \text{ and } w'\mathbbm{1} = 1$$ which can be solved by using a Lagrangian function: $L(w,\lambda) =  \frac{1}{2}w'\Sigma w - \lambda((\mu-R_0\mathbbm{1})'w - \mu_p-R_0)$. 
    \\
    We derive with respect to the two parameters:
    \begin{align*}
      &\frac{\partial L}{\partial w} = \Sigma w -\lambda (\mu-R_0\mathbbm{1}) = 0 \\
      &\frac{\partial L}{\partial \lambda} = (\mu-R_0\mathbbm{1})'w - \mu_p-R_0 = 0 
    \end{align*}
    And, thus: $w = \lambda\Sigma^{-1}(\mu-R_0\mathbbm{1})$. $\lambda$ can be calculated using the second constraint (the sum of weights equals $1$, only risky-assets in the portfolio). So, we obtain:
    $$ \lambda = (\mathbbm{1}'\Sigma^{-1}(\mu-R_0\mathbbm{1}))^{-1}$$
    
    Using the course notations, with $A = \mathbbm{1}'\Sigma^{-1}\mu$ and $B=\mathbbm{1}\Sigma^{-1}\mathbbm{1}$, we end up with:
    $$ w_t = \frac{1}{B-AR_0}\Sigma^{-1}(\mu-R_0\mathbbm{1})$$
    
    In our case, we have $3$ risky assets. The covariance matrix equals:
    $$ \Sigma = \left[ {\begin{array}{ccc}
      0.0225  &  0.0075  &  0.0090\\
      0.0075  &  0.0625  &  0.0150\\
      0.0090  &  0.0150  &  0.0900
    \end{array} } \right]$$ since $Cov(i,j) = Corr(i,j)\sigma_i\sigma_j$. Because $A = 54.4444$ and $B = 5.4778$, so the tangency portfolio has the following weights: $$ w_t = (0.4407, 0.2886, 0.2707)$$ 
    The tangency mean and standard deviation are $\mu_t = w_t'\mu = 11.22\%$ and $\sigma_t = \sqrt{w_t'\Sigma w_t} = 15.02\%$. Its Sharpe Ratio is $SR_t = \frac{\mu_t - R_0}{\sigma_t} = 0.4140$. 
    \item To find the zero beta portfolio $w_z$, the portfolio has to satisfy $w_z'\Sigma w_t = 0$ and is of the form $w = \lambda\Sigma^{-1}\mathbbm{1}+ \gamma\Sigma^{-1}\mu$ since it is also a mean-variance portfolio. Using the previous results, we can rewrite it as:
    \begin{align*}
      0 &= w_z'\Sigma(\frac{1}{B-AR_0}\Sigma^{-1}(\mu-R_0\mathbbm{1})) \\
      &= \frac{1}{B-AR_0}w_z'(\mu-R_0\mathbbm{1}) \\
      &= \frac{1}{B-AR_0}(w_z'\mu - R_0w_z'\mathbbm{1}) \\
      &= \frac{1}{B-AR_0}(\mu_z - R_0)
    \end{align*} because the portfolio is invested in risky assets only, so $w_z'\mathbbm{1} = 1$. And we require $\mu_z = R_0 = 5\%$. 
    \\
    We know $\mu_z$, so we can now compute the parameters $\lambda = \frac{C-\mu_zB}{AC-B^2}$ and $\gamma = \frac{\mu_zA-B}{AC-B^2}$ with $C = \mu'\Sigma^{-1}\mu$. 
    We obtain $C = 0.5830$ and so $\lambda= 0.1779$ and $\gamma = -1.5858$. 
    Finally, the zero beta portfolio is: 
    $$ w_z = (1.9099, -0.2748, -0.6351)$$
    And the corresponding mean, standard deviation and Sharpe ratio: 
    $$ \mu_z = 5\%, \quad \sigma_z = 31.41\%, \quad SR_z = 0$$
  \end{itemize}

\end{exercise}
  
\end{document}
