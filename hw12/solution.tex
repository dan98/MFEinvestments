\documentclass[10pt]{article}

\usepackage[T1]{fontenc}
\usepackage[utf8]{inputenc}
%\usepackage{beton}
%\usepackage{ccfonts}
%\usepackage{concrete}
\usepackage{concmath}
\usepackage{eulervm}
\usepackage{amsmath,amsthm,amssymb}
\usepackage{mathtools}
\usepackage{multicol}
\usepackage{marginnote}
\usepackage{pgfplots}
\usepackage{float}
\usepackage{hyperref}
\usepackage{bbm}
\usepackage{booktabs}
\usepackage{xcolor-solarized}
\usepackage{xcolor}
\usepackage{accents}
\pgfplotsset{compat=1.5}

\usepackage{listings}
\usepackage{xcolor}
\definecolor{codegreen}{rgb}{0,0.6,0}
\definecolor{codegray}{rgb}{0.5,0.5,0.5}
\definecolor{codepurple}{rgb}{0.58,0,0.82}
\definecolor{backcolour}{rgb}{0.95,0.95,0.92}
\lstdefinestyle{mystyle}{
    backgroundcolor=\color{backcolour},   
    commentstyle=\color{codegreen},
    keywordstyle=\color{magenta},
    numberstyle=\tiny\color{codegray},
    stringstyle=\color{codepurple},
    basicstyle=\ttfamily\footnotesize,
    breakatwhitespace=false,         
    breaklines=true,                 
    captionpos=b,                    
    keepspaces=true,                 
    numbers=left,                    
    numbersep=5pt,                  
    showspaces=false,                
    showstringspaces=false,
    showtabs=false,                  
    tabsize=2
}

\lstset{language=Python, style=mystyle}

\usepackage{mathtools}

\usepackage{wasysym}
\usepackage[margin=1.5in]{geometry} 
\usepackage{enumerate}
\index{\usepackage}\usepackage{multicol}

\newcommand{\N}{\mathbf{N}}
\newcommand{\Z}{\mathbb{Z}}

\newcommand{\R}{\mathbf{R}}
\newcommand{\C}{\mathbf{C}}
\newcommand{\Pbb}{\mathbb{P}}
\newcommand{\Fcal}{\mathcal{F}}
\newcommand{\Lcal}{\mathcal{L}}
\newcommand{\Acal}{\mathcal{A}}
\newcommand{\Ecal}{\mathcal{E}}
\newcommand{\Ebb}{\mathbb{E}}
\newcommand{\Qbb}{\mathbb{Q}}


\renewcommand{\mathbf}{\mathbold}

\newenvironment{theorem}[2][Theorem]{\begin{trivlist}
  \item[\hskip \labelsep {\bfseries #1}\hskip \labelsep {\bfseries #2.}]}{\end{trivlist}}
\newenvironment{lemma}[2][Lemma]{\begin{trivlist}
  \item[\hskip \labelsep {\bfseries #1}\hskip \labelsep {\bfseries #2.}]}{\end{trivlist}}
\newenvironment{exercise}[2][Exercise]{\begin{trivlist}
  \item[\hskip \labelsep {\bfseries #1}\hskip \labelsep {\bfseries #2.}]}{\end{trivlist}}
\newenvironment{reflection}[2][Reflection]{\begin{trivlist}
  \item[\hskip \labelsep {\bfseries #1}\hskip \labelsep {\bfseries #2.}]}{\end{trivlist}}
\newenvironment{proposition}[2][Proposition]{\begin{trivlist}
  \item[\hskip \labelsep {\bfseries #1}\hskip \labelsep {\bfseries #2.}]}{\end{trivlist}}
\newenvironment{corollary}[2][Corollary]{\begin{trivlist}
  \item[\hskip \labelsep {\bfseries #1}\hskip \labelsep {\bfseries #2.}]}{\end{trivlist}}

\newenvironment{definition}[2][Definition]{\begin{trivlist}
  \item[\hskip \labelsep {\bfseries #1}\hskip \labelsep {\bfseries #2.}]}{\end{trivlist}}

\definecolor{solar}{rgb}{0.9960, 0.9960, 0.9647}

\begin{document}
  \pagecolor{solar}
	
  \renewcommand{\qedsymbol}{\smiley}
	\title{Investments Class \\ Problem set 11}
	\author{Daniel Grosu, William Martin, Denis Steffen}
		
\maketitle

\begin{exercise} 1
  To compute a forward price of a forward contract on a stock, one can easily doe the cash \& carry argument by replicating the contract buying the stock and borrowing money. We can use a general expression for the forward price $F_t(T)$ at date $t$ for maturity $T$. 

  To do so, we borrow $F_t(T)e^{-r(T-t)}$ at the bank at date $t$ and so we will need to pay back $F_t(T)$ at maturity. We also borrow the discounted amount of dividend $\delta e^{-r(\theta-t)}$ for a dividend $\delta$ paid at date $\theta \in (t,T]$. Then we buy the stock at date t. Thus the cash flows at date $t$ are $$ -S_t + F_t(T)e^{-r(T-t)} + \delta e^{-r(\theta-t)}$$
  And we replicate properly the forward contract, as the cash flows are $0$ at date $\theta$ and $S_T - F_t(T)$ at maturity. 

  The cash flows at date $t$ must be $0$ to avoid an arbitrage opportunity between our strategy and buying the contract, thus: 
  $$ F_t(T) = e^{r(T-t)}(S_t - \delta e^{-r(\theta-t)})$$

  \begin{itemize}
    \item[(a)]
    To compute the forward price of the three-month contract, we have $r = 0.06$, $S_t = 24.50$, $\delta = 0.50$, $T = 3/12$, $t=0$ and $\theta = 2/12$. 
    So: 
    $$ F_0(3/12) = e^{0.06(3/12)}(24.50 - 0.50 e^{-0.06(2/12)}) = 26.36776 $$
     
    \item[(b)] 
    In this case, we have to change the value of $t = 1/12$ and $S_t = 23.50$, so:
    $$F_1(3/12) = e^{0.06(2/12)}(23.50 - 0.50e^{0.06(1/12)}) = 23.23367$$
    \item[(c)]  
    Finally, we have with the parameters $S_t = 24.50$ and $\delta = 1.00$: 
    $$ F_1(3/12) = e^{0.06(2/12)}(24.50 - 1.00e^{-0.06(1/12)}) = 23.73122 $$
  \end{itemize}


\end{exercise}

\end{document}



\appendix


