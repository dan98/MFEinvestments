\documentclass[10pt]{article}

\usepackage[T1]{fontenc}
\usepackage[utf8]{inputenc}
%\usepackage{beton}
%\usepackage{ccfonts}
%\usepackage{concrete}
\usepackage{concmath}
\usepackage{eulervm}
\usepackage{amsmath,amsthm,amssymb}
\usepackage{mathtools}
\usepackage{multicol}
\usepackage{marginnote}
\usepackage{pgfplots}
\usepackage{float}
\usepackage{hyperref}
\usepackage{bbm}
\usepackage{booktabs}
\usepackage{xcolor-solarized}
\usepackage{xcolor}
\usepackage{accents}
\pgfplotsset{compat=1.5}

\usepackage{listings}
\usepackage{xcolor}
\definecolor{codegreen}{rgb}{0,0.6,0}
\definecolor{codegray}{rgb}{0.5,0.5,0.5}
\definecolor{codepurple}{rgb}{0.58,0,0.82}
\definecolor{backcolour}{rgb}{0.95,0.95,0.92}
\lstdefinestyle{mystyle}{
    backgroundcolor=\color{backcolour},   
    commentstyle=\color{codegreen},
    keywordstyle=\color{magenta},
    numberstyle=\tiny\color{codegray},
    stringstyle=\color{codepurple},
    basicstyle=\ttfamily\footnotesize,
    breakatwhitespace=false,         
    breaklines=true,                 
    captionpos=b,                    
    keepspaces=true,                 
    numbers=left,                    
    numbersep=5pt,                  
    showspaces=false,                
    showstringspaces=false,
    showtabs=false,                  
    tabsize=2
}

\lstset{language=Python, style=mystyle}

\usepackage{mathtools}

\usepackage{wasysym}
\usepackage[margin=1.5in]{geometry} 
\usepackage{enumerate}
\index{\usepackage}\usepackage{multicol}

\newcommand{\N}{\mathbf{N}}
\newcommand{\Z}{\mathbb{Z}}

\newcommand{\R}{\mathbf{R}}
\newcommand{\C}{\mathbf{C}}
\newcommand{\Pbb}{\mathbb{P}}
\newcommand{\Fcal}{\mathcal{F}}
\newcommand{\Lcal}{\mathcal{L}}
\newcommand{\Acal}{\mathcal{A}}
\newcommand{\Ecal}{\mathcal{E}}
\newcommand{\Ebb}{\mathbb{E}}
\newcommand{\Qbb}{\mathbb{Q}}


\renewcommand{\mathbf}{\mathbold}

\newenvironment{theorem}[2][Theorem]{\begin{trivlist}
  \item[\hskip \labelsep {\bfseries #1}\hskip \labelsep {\bfseries #2.}]}{\end{trivlist}}
\newenvironment{lemma}[2][Lemma]{\begin{trivlist}
  \item[\hskip \labelsep {\bfseries #1}\hskip \labelsep {\bfseries #2.}]}{\end{trivlist}}
\newenvironment{exercise}[2][Exercise]{\begin{trivlist}
  \item[\hskip \labelsep {\bfseries #1}\hskip \labelsep {\bfseries #2.}]}{\end{trivlist}}
\newenvironment{reflection}[2][Reflection]{\begin{trivlist}
  \item[\hskip \labelsep {\bfseries #1}\hskip \labelsep {\bfseries #2.}]}{\end{trivlist}}
\newenvironment{proposition}[2][Proposition]{\begin{trivlist}
  \item[\hskip \labelsep {\bfseries #1}\hskip \labelsep {\bfseries #2.}]}{\end{trivlist}}
\newenvironment{corollary}[2][Corollary]{\begin{trivlist}
  \item[\hskip \labelsep {\bfseries #1}\hskip \labelsep {\bfseries #2.}]}{\end{trivlist}}

\newenvironment{definition}[2][Definition]{\begin{trivlist}
  \item[\hskip \labelsep {\bfseries #1}\hskip \labelsep {\bfseries #2.}]}{\end{trivlist}}

\definecolor{solar}{rgb}{0.9960, 0.9960, 0.9647}

\begin{document}
  \pagecolor{solar}
	
  \renewcommand{\qedsymbol}{\smiley}
	\title{Investments Class \\ Problem set 11}
	\author{Daniel Grosu, William Martin, Denis Steffen}
		
\maketitle

\begin{exercise} 1
  To compute a forward price of a forward contract on a stock, one can easily doe the cash \& carry argument by replicating the contract buying the stock and borrowing money. We can use a general expression for the forward price $F_t(T)$ at date $t$ for maturity $T$. 

  To do so, we borrow $F_t(T)e^{-r(T-t)}$ at the bank at date $t$ and so we will need to pay back $F_t(T)$ at maturity. We also borrow the discounted amount of dividend $\delta e^{-r(\theta-t)}$ for a dividend $\delta$ paid at date $\theta \in (t,T]$. Then we buy the stock at date t. Thus the cash flows at date $t$ are $$ -S_t + F_t(T)e^{-r(T-t)} + \delta e^{-r(\theta-t)}$$
  And we replicate properly the forward contract, as the cash flows are $0$ at date $\theta$ and $S_T - F_t(T)$ at maturity. 

  The cash flows at date $t$ must be $0$ to avoid an arbitrage opportunity between our strategy and buying the contract, thus: 
  $$ F_t(T) = e^{r(T-t)}(S_t - \delta e^{-r(\theta-t)})$$

  \begin{itemize}
    \item[(a)]
    To compute the forward price of the three-month contract, we have $r = 0.06$, $S_t = 24.50$, $\delta = 0.50$, $T = 3/12$, $t=0$ and $\theta = 2/12$. 
    So: 
    $$ F_0(3/12) = e^{0.06(3/12)}(24.50 - 0.50 e^{-0.06(2/12)}) = 26.36776 $$
     
    \item[(b)] 
    In this case, we have to change the value of $t = 1/12$ and $S_t = 23.50$, so:
    $$F_1(3/12) = e^{0.06(2/12)}(23.50 - 0.50e^{0.06(1/12)}) = 23.23367$$
    \item[(c)]  
    Finally, we have with the parameters $S_t = 24.50$ and $\delta = 1.00$: 
    $$ F_1(3/12) = e^{0.06(2/12)}(24.50 - 1.00e^{-0.06(1/12)}) = 23.73122 $$
  \end{itemize}


\end{exercise}

\newpage

\begin{exercise} 2
\end{exercise}

a) The 6-month forward futures prices of the SPX and STOXX can be computed using the formula for the  forward price

\begin{align*}
	F_t^T = S_te^{(r + v - c)(T - t)}
\end{align*}

Which gives us the following

\begin{align*}
	&F_{SPX}^{6months} = 1265 e^{(0.03)\frac{1}{2}} = 1284.12 \\
	&F_{STOXX}^{6months} = 3671 e^{(0.05)\frac{1}{2}} = 3763.93
\end{align*}

The currency exchange rate between the dollar (domestic) and the euro (foreign) can be computed using the formula for the covered interest parity (CIP):

\begin{align*}
	F_{usd/eur}^{6months} = 1.28 e^{(0.03 - 0.05)\frac{1}{2}} = 1.267
\end{align*}

b) The 6-month futures of the STOXX is quoted at EUR 3782. This is not correctly priced since our calculations gave us EUR 3763.93 (it is actually overpriced). Since we assume that borrowing or lending in euros or dollars is not allowed,  the only actions we can undertake is going short and long on both indexes. The following arbitrage strategy can be undertaken to profit from this

\begin{enumerate}
	\item Sell 1 unit of SPX index at $t = 0$. Balance: USD $+1265$, EUR $+0$
	\item Convert proceedings to EUR at $t=0$ using the current exchange rate of USD 1.28 / EUR. Balance: USD $+0$, EUR $+988.28$
	\item Buy $\frac{988.28}{3671} = 0.269$ units of STOXX index at $t = 0$ such that the total balance in EUR is 0. Balance: USD $+0$,  EUR $+0$
\end{enumerate}

At $t = 0$, we thus have USD $+0$, EUR $+0$ but  0.269 units of STOXX. We continue our arbitrage strategy

\begin{enumerate}
	\setcounter{enumi}{3}
	\item Short 0.269 units of STOXX futures, which is quoted at EUR $3782$.  Balance: USD $0$, EUR $1018.16$
	\item Convert the proceedings back to USD using the currency forwards computed in part $a$ (1.267). Balance: USD $+1290.01$, EUR $+0$
	\item We buy back the unite of SPX index sold initially at the futures price of USD $1284.12$. Balance: USD $5.89$, EUR $+0$  
\end{enumerate}

We have thus made a profite of USD $5.89$ using this strategy and taking advantage of the overpriced futures of the STOXX index. 

\end{document}



\appendix


