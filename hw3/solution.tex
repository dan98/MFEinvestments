\documentclass[10pt]{article}

\usepackage[T1]{fontenc}
\usepackage[utf8]{inputenc}
%\usepackage{beton}
%\usepackage{ccfonts}
%\usepackage{concrete}
\usepackage{concmath}
\usepackage{eulervm}
\usepackage{amsmath,amsthm,amssymb}
\usepackage{mathtools}
\usepackage{multicol}
\usepackage{marginnote}
\usepackage{pgfplots}
\usepackage{float}
\usepackage{hyperref}
\usepackage{bbm}
\usepackage{booktabs}
\pgfplotsset{compat=1.5}

\usepackage{listings}
\usepackage{xcolor}
\definecolor{codegreen}{rgb}{0,0.6,0}
\definecolor{codegray}{rgb}{0.5,0.5,0.5}
\definecolor{codepurple}{rgb}{0.58,0,0.82}
\definecolor{backcolour}{rgb}{0.95,0.95,0.92}
\lstdefinestyle{mystyle}{
    backgroundcolor=\color{backcolour},   
    commentstyle=\color{codegreen},
    keywordstyle=\color{magenta},
    numberstyle=\tiny\color{codegray},
    stringstyle=\color{codepurple},
    basicstyle=\ttfamily\footnotesize,
    breakatwhitespace=false,         
    breaklines=true,                 
    captionpos=b,                    
    keepspaces=true,                 
    numbers=left,                    
    numbersep=5pt,                  
    showspaces=false,                
    showstringspaces=false,
    showtabs=false,                  
    tabsize=2
}

\lstset{language=Python, style=mystyle}

\usepackage{mathtools}

\usepackage{wasysym}
\usepackage[margin=1.5in]{geometry} 
\usepackage{enumerate}
\index{\usepackage}\usepackage{multicol}

\newcommand{\N}{\mathbf{N}}
\newcommand{\Z}{\mathbb{Z}}

\newcommand{\R}{\mathbf{R}}
\newcommand{\C}{\mathbf{C}}
\newcommand{\Pbb}{\mathbb{P}}
\newcommand{\Fcal}{\mathcal{F}}
\newcommand{\Lcal}{\mathcal{L}}
\newcommand{\Acal}{\mathcal{A}}
\newcommand{\Ecal}{\mathcal{E}}
\newcommand{\Ebb}{\mathbb{E}}
\newcommand{\Qbb}{\mathbb{Q}}


\renewcommand{\mathbf}{\mathbold}

\newenvironment{theorem}[2][Theorem]{\begin{trivlist}
  \item[\hskip \labelsep {\bfseries #1}\hskip \labelsep {\bfseries #2.}]}{\end{trivlist}}
\newenvironment{lemma}[2][Lemma]{\begin{trivlist}
  \item[\hskip \labelsep {\bfseries #1}\hskip \labelsep {\bfseries #2.}]}{\end{trivlist}}
\newenvironment{exercise}[2][Exercise]{\begin{trivlist}
  \item[\hskip \labelsep {\bfseries #1}\hskip \labelsep {\bfseries #2.}]}{\end{trivlist}}
\newenvironment{reflection}[2][Reflection]{\begin{trivlist}
  \item[\hskip \labelsep {\bfseries #1}\hskip \labelsep {\bfseries #2.}]}{\end{trivlist}}
\newenvironment{proposition}[2][Proposition]{\begin{trivlist}
  \item[\hskip \labelsep {\bfseries #1}\hskip \labelsep {\bfseries #2.}]}{\end{trivlist}}
\newenvironment{corollary}[2][Corollary]{\begin{trivlist}
  \item[\hskip \labelsep {\bfseries #1}\hskip \labelsep {\bfseries #2.}]}{\end{trivlist}}

\newenvironment{definition}[2][Definition]{\begin{trivlist}
  \item[\hskip \labelsep {\bfseries #1}\hskip \labelsep {\bfseries #2.}]}{\end{trivlist}}

\begin{document}
	
  \renewcommand{\qedsymbol}{\smiley}
	\title{Investments Class \\ Problem set 3}
	\author{Daniel Grosu, William Martin, Denis Steffen}
	
	\maketitle

  \begin{exercise}{1}(Efficient portfolios)
    \begin{itemize}
      \item
    The return on the portfolio is linear in the returns of the risky assets and
    the risk-free return:

    \begin{align*}
      \mu_P &= \mu^T \omega + (1 - \mathbbm{1}^T \omega) R_f \\
      \mu_P - R_f &= \mu^T \omega +  - \mathbbm{1}^T \omega R_f \\
      \mu_P - R_f &= (\mu - R_f\mathbbm{1})^T \omega
    \end{align*}
    where $\mathbbm{1}$ is the $N \times 1$ vector of ones. The maximization is
    then formulated in terms of the Lagrangian with no constraints as follows
    \begin{align*}
      \Lcal = (R_f + (\mu - \mathbbm{1}R_f)^T \omega) - \gamma \frac 1 2 \omega^T \Sigma \omega.
    \end{align*}
    
    The FOC condition with respect to $\omega$ reads as:
    \begin{align*}
      \frac{\partial \Lcal}{\partial \omega} = \mu - \mathbbm{1} R_f - \gamma \Sigma \omega = 0
    \end{align*}
    and solving for $\omega$ yields the following optimal $\omega^*$:
    \begin{align*}
      \omega^* = \gamma^{-1}\Sigma^{-1} \left( \mu - R_f \mathbbm{1} \right)
    \end{align*}
    The $\omega^*$ is optimal since the Hessian $\frac{\partial^2 \Lcal}{\partial \omega\partial \omega^\top} =
    -\Sigma$ which is negative semi-definite.
      \begin{align*}
        cov \left[R, R_p \right] &= cov \left[ R, R_f + (\omega^*)^T(R - R_f\mathbbm{1}) \right] \\
        &= cov \left[ R, (\omega^*)^T R \right]a \\
        &= cov \left[ R - \mu, R - \mu \right] \omega^* \\
        &= var \left[ R, R \right] \omega^* \\
        &= \Sigma \omega^* \\
        &= \Sigma \gamma^{-1} \Sigma^{-1} (\mu - R_f \mathbbm{1}) \\
        &=\gamma^{-1}(\mu - R_f \mathbbm{1}) \\
        &\Downarrow \\
          \gamma cov\left[ R, R_p \right]&= \mu - R_f\mathbbm{1} \;\;\; (\star)
      \end{align*}
      \item
        \begin{align*}
          \mu_P = \Ebb\left[ R_p \right] &= \Ebb\left[ R_f + \omega^T(R - R_f\mathbbm{1}) \right] \\
          &= R_f + \omega^T ( \mu - R_f \mathbbm{1}) \\                                         &= \omega^T \mu + (1 - \omega^T \mathbbm{1}) R_f \\
          &= \omega^T(\mu - \mathbbm{1}R_f) + R_f \\
          &\stackrel{\star}{=} \omega^T\gamma cov\left[ R, R_p \right] + R_f \\
          &= \gamma cov \left[ \omega^TR, R_p \right]+ R_f \\
          &= \gamma cov \left[ R_p, R_p \right] + R_f\\
          &= \gamma \sigma^2_P  + R_f\\
          &\Downarrow \\
          \mu_P - R_f &= \gamma \sigma_P^2\\
          &\Downarrow \\
          \gamma &= \frac{\mu_P - R_f}{\sigma_P^2}\\
          & \Downarrow \\
          (\mu - R_f \mathbbm{1}) &\stackrel{\star}{=} \gamma cov\left[ R, R_p \right] = \frac{cov\left[ R, R_p \right]}{\sigma_P^2} (\mu_P - R_f) = \beta_{P} (\mu_P - R_f)
          \end{align*}
          Where $\beta_{P} = (\beta_{1, P}, \ldots, \beta_{N, P})^T$.

          \item
            A linear regression model is
            \begin{align*}
              Y = X_1 \beta_1 + \cdots + X_N \beta_N + \epsilon
            \end{align*}
            with the following assumptions:
            \begin{enumerate}[\textbf{A}1.]
              \item \textbf{Linearity} between $Y$ and $X_i$.
              \item \textbf{Full rank}  of $X$.
              \item \textbf{Exogeneity}: $\Ebb\left[ \epsilon \mid X_1,
                  \cdots, X_N \right] = 0$.
              \item \textbf{Homoskedasticity and nonautocorrelation}: 
                $cov(\epsilon_i, \epsilon_j | X_i, X_j) = 0 \;\; \forall i \neq j$
                and $var(\epsilon_i \mid X_1, \ldots, X_N) = \sigma^2$.
              \item \textbf{Normality of the errors}: $\epsilon \mid X_1,
                \ldots, X_N \sim N(0, \sigma^2)$.
            \end{enumerate}

            

    \end{itemize}
  \end{exercise}

\begin{exercise}{2}(Portfolio math)
   First we show that a convex combination of two minimum variance frontier portfolios $w_a$ and $w_b$ is a minimum variance frontier portfolio.

   $w$ is a minimum variance frontier portfolio if it solves the optimization problem and thus can be written as: 
   $$ w = \lambda\Sigma^{-1}\mathbbm{1}+ \gamma\Sigma^{-1}\mu$$ with $\lambda = \frac{C- \mu_pB}{\Delta}$ and $\gamma = \frac{\mu_pA-B}{\Delta}$. 

   Consider now two arbitrary minimum variance porfolios $w_a$ and $w_b$ with returns $\mu_a$, $\mu_b$ respectively and write $w = \alpha w_a + (1-\alpha)w_b$. 
   Then \begin{align*}
    w &= \alpha (\lambda_a\Sigma^{-1}\mathbbm{1} + \gamma_a\Sigma^{-1}\mu) + (1-\alpha)(\lambda_b\Sigma^{-1}\mathbbm{1} + \gamma_b\Sigma^{-1}\mu) \\
    &= (\alpha \lambda_a + (1-\alpha) \lambda_b)\Sigma^{-1}\mathbbm{1} + (\alpha \gamma_a + (1-\alpha) \gamma_b)\Sigma^{-1}\mu
   \end{align*} 
   But $A,B,C,D$ do not depend on the portfolio but only on the available assets so they do not change between $w_a$ and $w_b$. So we obtain:
   $$ w = \frac{C- (\alpha\mu_a+(1-\alpha)\mu_b)B}{\Delta}\Sigma^{-1}\mathbbm{1} + \frac{(\alpha\mu_p + (1-\alpha)\mu_b)A-B}{\Delta}\Sigma^{-1}\mu$$

  The return of $w$ equals $\alpha\mu_a+(1-\alpha)\mu_b$ and its variance is $\alpha^2 w_a^\top\Sigma w_a + (1-\alpha)^2 w_b^\top\Sigma w_b + 2\alpha(1-\alpha)w_a^\top\Sigma w_b$

  Finally, $w$ is of the form of a minimum variance frontier portfolio solving the following optimization problem: 
  $$ \min_{w} \frac{1}{2} w^\top\Sigma w \text{ such that } w^\top\mu = \alpha\mu_a+(1-\alpha)\mu_b \text{ and } w^\top\mathbbm{1} = 1$$
  
  Conversely, if we have the minimum variance portfolio $w$ and want to write it as a combination of two different portfolios $w_a$ and $w_b$ (with return $\mu_a$ and $\mu_b$), we can define $\alpha = \frac{\mu_p-\mu_b}{\mu_a-\mu_b}, 1- \alpha = \frac{\mu_a - \mu_p}{\mu_a - \mu_b}$. 
  Thus, \begin{align*}
    \alpha w_a + (1-\alpha)w_b &= \frac{C- (\alpha\mu_a+(1-\alpha)\mu_b)B}{\Delta}\Sigma^{-1}\mathbbm{1} + \frac{(\alpha\mu_p + (1-\alpha)\mu_b)A-B}{\Delta}\Sigma^{-1}\mu \\
    &= \frac{C- (\mu_p)B}{\Delta}\Sigma^{-1}\mathbbm{1} + \frac{(\mu_p)A-B}{\Delta}\Sigma^{-1}\mu = w
  \end{align*} because $ \frac{\mu_p-\mu_b}{\mu_a-\mu_b}\mu_a + \frac{\mu_a - \mu_p}{\mu_a - \mu_b}\mu_b = \frac{\mu_p(\mu_a-\mu_b)-\mu_a\mu_b + \mu_a\mu_b}{\mu_a-\mu_b} = \mu_p$
\\
  This proves that any minimum variance frontier portfolio can be replicated by a convex combination of two minimum variance frontier portfolios. 

  For the second part of the exercise, we follow the hint. The expected return of the convex combination of the two portfolios is: $ \Ebb[Y] = wR + (1-w)R_{\text{min}}$. In addition, the variance of this portfolio is given by:
  $$ Var(Y) = w^2Var(R) + (1-w)^2Var(R_{\text{min}}) + 2w(1-w)Cov(R,R_{\text{min}})$$
  Since the global minimum-variance portfolio has the minimal variance among all portfolios, we can see that $Var(Y) \geq Var(R_{\text{min}})$ and this value is achieved when $w = 0$. 

  To find the minimum algebraically, we can differentiate the variance with respect to $w$: $$ \frac{\partial Var(Y)}{\partial w } = 2wVar(R) - 2(1-w)Var(R_{\text{min}})+2(1-2w)Cov(R,R_{\text{min}})$$ and equate it with $0$:
  $$ w(2Var(R) + 2Var(R_{\text{min}})-4Cov(R,R_{\text{min}})) = 2Var(R_{\text{min}})-2Cov(R,R_{\text{min}})$$ that leads to: 
  $$ \hat{w} = \frac{Var(R_{\text{min}})-Cov(R,R_{\text{min}})}{Var(R)+Var(R_{\text{min}})-Cov(R,R_{\text{min}})}$$
  But to be the unique minimum, $\hat{w}$ needs to be $0$ as we saw previously. 

  Thus, $ Var(R_{\text{min}}) = Cov(R,R_{\text{min}})$.

\end{exercise}
  
\end{document}

