\documentclass[10pt]{article}

\usepackage[T1]{fontenc}
\usepackage[utf8]{inputenc}
%\usepackage{beton}
%\usepackage{ccfonts}
%\usepackage{concrete}
\usepackage{concmath}
\usepackage{eulervm}
\usepackage{amsmath,amsthm,amssymb}
\usepackage{mathtools}
\usepackage{multicol}
\usepackage{marginnote}
\usepackage{pgfplots}
\usepackage{float}
\usepackage{hyperref}
\usepackage{bbm}
\usepackage{booktabs}
\usepackage{xcolor-solarized}
\usepackage{xcolor}
\usepackage{accents}
\pgfplotsset{compat=1.5}

\usepackage{listings}
\usepackage{xcolor}
\definecolor{codegreen}{rgb}{0,0.6,0}
\definecolor{codegray}{rgb}{0.5,0.5,0.5}
\definecolor{codepurple}{rgb}{0.58,0,0.82}
\definecolor{backcolour}{rgb}{0.95,0.95,0.92}
\lstdefinestyle{mystyle}{
    backgroundcolor=\color{backcolour},   
    commentstyle=\color{codegreen},
    keywordstyle=\color{magenta},
    numberstyle=\tiny\color{codegray},
    stringstyle=\color{codepurple},
    basicstyle=\ttfamily\footnotesize,
    breakatwhitespace=false,         
    breaklines=true,                 
    captionpos=b,                    
    keepspaces=true,                 
    numbers=left,                    
    numbersep=5pt,                  
    showspaces=false,                
    showstringspaces=false,
    showtabs=false,                  
    tabsize=2
}

\lstset{language=Python, style=mystyle}

\usepackage{mathtools}

\usepackage{wasysym}
\usepackage[margin=1.5in]{geometry} 
\usepackage{enumerate}
\index{\usepackage}\usepackage{multicol}

\newcommand{\N}{\mathbf{N}}
\newcommand{\Z}{\mathbb{Z}}

\newcommand{\R}{\mathbf{R}}
\newcommand{\C}{\mathbf{C}}
\newcommand{\Pbb}{\mathbb{P}}
\newcommand{\Fcal}{\mathcal{F}}
\newcommand{\Lcal}{\mathcal{L}}
\newcommand{\Acal}{\mathcal{A}}
\newcommand{\Ecal}{\mathcal{E}}
\newcommand{\Ebb}{\mathbb{E}}
\newcommand{\Qbb}{\mathbb{Q}}


\renewcommand{\mathbf}{\mathbold}

\newenvironment{theorem}[2][Theorem]{\begin{trivlist}
  \item[\hskip \labelsep {\bfseries #1}\hskip \labelsep {\bfseries #2.}]}{\end{trivlist}}
\newenvironment{lemma}[2][Lemma]{\begin{trivlist}
  \item[\hskip \labelsep {\bfseries #1}\hskip \labelsep {\bfseries #2.}]}{\end{trivlist}}
\newenvironment{exercise}[2][Exercise]{\begin{trivlist}
  \item[\hskip \labelsep {\bfseries #1}\hskip \labelsep {\bfseries #2.}]}{\end{trivlist}}
\newenvironment{reflection}[2][Reflection]{\begin{trivlist}
  \item[\hskip \labelsep {\bfseries #1}\hskip \labelsep {\bfseries #2.}]}{\end{trivlist}}
\newenvironment{proposition}[2][Proposition]{\begin{trivlist}
  \item[\hskip \labelsep {\bfseries #1}\hskip \labelsep {\bfseries #2.}]}{\end{trivlist}}
\newenvironment{corollary}[2][Corollary]{\begin{trivlist}
  \item[\hskip \labelsep {\bfseries #1}\hskip \labelsep {\bfseries #2.}]}{\end{trivlist}}

\newenvironment{definition}[2][Definition]{\begin{trivlist}
  \item[\hskip \labelsep {\bfseries #1}\hskip \labelsep {\bfseries #2.}]}{\end{trivlist}}

\definecolor{solar}{rgb}{0.9960, 0.9960, 0.9647}

\begin{document}
  \pagecolor{solar}
	
  \renewcommand{\qedsymbol}{\smiley}
	\title{Investments Class \\ Problem set 10}
	\author{Daniel Grosu, William Martin, Denis Steffen}
		
\maketitle

\begin{exercise}{1. Fund performance and fees}
\end{exercise}

\newpage

\begin{exercise}{2. Closed-end funds}
\end{exercise}

a) Let $V_t$ be the net asset value (NAV) of the fund at time \textit{t}. At this moment, the fund has not yet paid the fund investors and management fees. Therefore, the NAV at time $t+1$, that is $V_{t+1}$ can be computed as follows, taking into account the return rate:

\begin{align*}
	V_{t+1} &= V_t(1 + R_{t+1}) - (f + \delta)V_t(1 + R_{t+1})\\
	&= (1 - f - \delta)V_t(1 + R_{t+1})
\end{align*}

b) We can start by expressing the NAV at time $t+2$ and $t+3$:

\begin{align*}
	V_{t+2} &= (1-f-\delta)V_{t+1}(1 + R_{t+2}\\
	&= (1-f-\delta)^2V_t(1 + R_{t+1})(1 + R_{t+2})\\
	V_{t+3} &= (1-f-\delta)^3 V_t (1+R_{t+1})(1+R_{t+2})(1+R_{t+3})\\
	&...
\end{align*}

Which can be generalized as follows

\begin{align*}
	V_{t+n} = (1-f-\delta)^n V_t \prod^n_{i=1}(1 + R_{t+i})
\end{align*}

Taking the expectation of $V_{t+n}$ gives us:

\begin{align*}
	\mathbb{E}[V_{t+n}] &= (1 - f - \delta)^n V_t \prod^n_{i=1} \mathbb{E}[1 + R_{t+i}]\\
	&= (1 - f - \delta)^n V_t \left(1 + R_f +\beta (\mu_M - R_f)\right)^n
\end{align*}

Where $\mathbb{E}[\epsilon_t] = 0$.



\end{document}



\appendix


