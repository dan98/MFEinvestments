\documentclass[10pt]{article}

\usepackage[T1]{fontenc}
\usepackage[utf8]{inputenc}
%\usepackage{beton}
%\usepackage{ccfonts}
%\usepackage{concrete}
\usepackage{concmath}
\usepackage{eulervm}
\usepackage{amsmath,amsthm,amssymb}
\usepackage{mathtools}
\usepackage{multicol}
\usepackage{marginnote}
\usepackage{pgfplots}
\usepackage{float}
\usepackage{hyperref}
\usepackage{bbm}
\usepackage{booktabs}
\usepackage{xcolor-solarized}
\usepackage{xcolor}
\pgfplotsset{compat=1.5}

\usepackage{listings}
\usepackage{xcolor}
\definecolor{codegreen}{rgb}{0,0.6,0}
\definecolor{codegray}{rgb}{0.5,0.5,0.5}
\definecolor{codepurple}{rgb}{0.58,0,0.82}
\definecolor{backcolour}{rgb}{0.95,0.95,0.92}
\lstdefinestyle{mystyle}{
    backgroundcolor=\color{backcolour},   
    commentstyle=\color{codegreen},
    keywordstyle=\color{magenta},
    numberstyle=\tiny\color{codegray},
    stringstyle=\color{codepurple},
    basicstyle=\ttfamily\footnotesize,
    breakatwhitespace=false,         
    breaklines=true,                 
    captionpos=b,                    
    keepspaces=true,                 
    numbers=left,                    
    numbersep=5pt,                  
    showspaces=false,                
    showstringspaces=false,
    showtabs=false,                  
    tabsize=2
}

\lstset{language=Python, style=mystyle}

\usepackage{mathtools}

\usepackage{wasysym}
\usepackage[margin=1.5in]{geometry} 
\usepackage{enumerate}
\index{\usepackage}\usepackage{multicol}

\newcommand{\N}{\mathbf{N}}
\newcommand{\Z}{\mathbb{Z}}

\newcommand{\R}{\mathbf{R}}
\newcommand{\C}{\mathbf{C}}
\newcommand{\Pbb}{\mathbb{P}}
\newcommand{\Fcal}{\mathcal{F}}
\newcommand{\Lcal}{\mathcal{L}}
\newcommand{\Acal}{\mathcal{A}}
\newcommand{\Ecal}{\mathcal{E}}
\newcommand{\Ebb}{\mathbb{E}}
\newcommand{\Qbb}{\mathbb{Q}}


\renewcommand{\mathbf}{\mathbold}

\newenvironment{theorem}[2][Theorem]{\begin{trivlist}
  \item[\hskip \labelsep {\bfseries #1}\hskip \labelsep {\bfseries #2.}]}{\end{trivlist}}
\newenvironment{lemma}[2][Lemma]{\begin{trivlist}
  \item[\hskip \labelsep {\bfseries #1}\hskip \labelsep {\bfseries #2.}]}{\end{trivlist}}
\newenvironment{exercise}[2][Exercise]{\begin{trivlist}
  \item[\hskip \labelsep {\bfseries #1}\hskip \labelsep {\bfseries #2.}]}{\end{trivlist}}
\newenvironment{reflection}[2][Reflection]{\begin{trivlist}
  \item[\hskip \labelsep {\bfseries #1}\hskip \labelsep {\bfseries #2.}]}{\end{trivlist}}
\newenvironment{proposition}[2][Proposition]{\begin{trivlist}
  \item[\hskip \labelsep {\bfseries #1}\hskip \labelsep {\bfseries #2.}]}{\end{trivlist}}
\newenvironment{corollary}[2][Corollary]{\begin{trivlist}
  \item[\hskip \labelsep {\bfseries #1}\hskip \labelsep {\bfseries #2.}]}{\end{trivlist}}

\newenvironment{definition}[2][Definition]{\begin{trivlist}
  \item[\hskip \labelsep {\bfseries #1}\hskip \labelsep {\bfseries #2.}]}{\end{trivlist}}

\definecolor{solar}{rgb}{0.9960, 0.9960, 0.9647}

\begin{document}
  \pagecolor{solar}
	
  \renewcommand{\qedsymbol}{\smiley}
	\title{Investments Class \\ Problem set 8}
	\author{Daniel Grosu, William Martin, Denis Steffen}
		
\maketitle

\begin{exercise}{1}(CAPM)
\end{exercise}
\begin{itemize}
  \item[(a)] By definition of the beta of an asset, we have: $$ \sigma_M^2 = \frac{Cov(r_A,r_M)}{\beta_A} = \frac{Corr(r_A,r_M)\sigma_M\sigma_A}{\beta_A}$$ and so $$ \sigma_M = \frac{Corr(r_A,r_M)\sigma_A}{\beta_A} = 0.2$$
  Thus we can deduce the beta of asset B: $$\beta_B = \frac{Corr(r_B,r_M)\sigma_M\sigma_B}{\sigma_M^2} = 0.75$$
  And, 
  $$Corr(r_C,r_M) = \frac{Cov(r_C,r_M)}{\sigma_C\sigma_M} = \frac{\beta_C\sigma_M}{\sigma_C}  = 0.8$$
  \item[(b)] Assuming the CAPM holds, we can know compute the market portfolio return using asset A: 
  $$ r_A = r_0 + \beta_A(r_M-r_0) \quad \Rightarrow \quad r_M = \frac{r_A-r_0}{\beta_A} + r_0 = 0.06$$ 
  And using the CAPM formula for assets B and C, we obtain: $ r_B = 0.05$ and $r_C = 0.076$.
  \item[(c)] We can see that asset D has a better expected return than all assets A, B and C, but has a lower standard deviation than C. So the security D will be over the market security line and is undervalued as it has greater return for less risk than C (that is on the line). 
  
  The alpha of D is in the following equation: 
  $$ r_D - r_0 = \alpha_D + \beta_D(r_M-r_0) \quad \Rightarrow \quad \alpha_D = r_D^e -\beta_D*r_M^e = 0.06 - 0.048 = 0.012.$$ 
  The systematic risk is given by: $\beta_D^2\sigma_M^2 = 0.0576$, and the residual risk equals the standard deviation minus the systematic risk: $\sigma_D^2-\beta_D^2\sigma_M^2 = 0.0208.$

  In addition, the Information Ratio is the ratio $IR = \frac{\alpha_D}{\sigma_D} =\frac{0.012}{0.08}= 0.15$
  \item[(d)] Finally, the optimal portfolio with risk-aversion $1$ is given by:
  $$ w = \Sigma^{-1}(\mu-r_0\mathbbm{1}_5)$$ with assets (A,B,C,M,D), $\mu = (r_A,r_B,r_C,r_M,r_D)^\top = (0.032,0.05,0.076,0.06,0.08)^\top$. So the covariance matrix $\Sigma$ is: 
  $$ \Sigma = \left[\begin{array}{ccccc}
    \sigma_A^2& 0 &0 &Cov(r_A,r_M)& 0\\
    0&\sigma_B^2 & 0 & Cov(r_B,r_M)&0 \\
    0&0&\sigma_C^2&Cov(r_C,r_M)&0\\
    Cov(r_A,r_M)&Cov(r_B,r_M)&Cov(r_C,r_M)&\sigma_M^2&Cov(r_D,r_M)\\
    0&0&0&Cov(r_D,r_M)&\sigma_D^2
  \end{array}\right]$$ 
  And with the given parameters: 
  $$ \Sigma = \left[\begin{array}{ccccc}
    0.0225 & 0 & 0 & 0.012 &0 \\
    0 & 0.0625 & 0 & 0.03 & 0 \\
    0 & 0 & 0.1225 & 0.056 & 0 \\
    0.012 & 0.03 & 0.056 & 0.04 & 0.048 \\
    0 & 0 & 0 & 0.048 & 0.0784 
  \end{array}\right]$$

  And the optimal portfolio weights are: 
 $$ w = (0.3562, -0.7581, 0.3392, 0.2579, 0.6074)$$
 The weight invested in the T-bills, that are our riskfree asset, is $w_0 = 1- \mathbbm{1}'w = 0.1974$.

 Now, we can compute the usual statistics for this portfolio. First, the expected return is $w'\mu + w_0r_0 = 0.0673$ and the standard deviation equals $\sqrt{w'\Sigma w} = 0.2175$. And the Sharpe Ratio is $ \frac{w'\mu - r_0}{\sqrt{w'\Sigma w}} = 0.2175$.
\end{itemize}
  
\end{document}



\appendix
