\documentclass[10pt]{article}

\usepackage[T1]{fontenc}
\usepackage[utf8]{inputenc}
%\usepackage{beton}
%\usepackage{ccfonts}
%\usepackage{concrete}
\usepackage{concmath}
\usepackage{eulervm}
\usepackage{amsmath,amsthm,amssymb}
\usepackage{mathtools}
\usepackage{multicol}
\usepackage{marginnote}
\usepackage{pgfplots}
\usepackage{float}
\usepackage{hyperref}
\usepackage{bbm}
\usepackage{booktabs}
\usepackage{xcolor-solarized}
\usepackage{xcolor}
\usepackage{accents}
\pgfplotsset{compat=1.5}

\usepackage{listings}
\usepackage{xcolor}
\definecolor{codegreen}{rgb}{0,0.6,0}
\definecolor{codegray}{rgb}{0.5,0.5,0.5}
\definecolor{codepurple}{rgb}{0.58,0,0.82}
\definecolor{backcolour}{rgb}{0.95,0.95,0.92}
\lstdefinestyle{mystyle}{
    backgroundcolor=\color{backcolour},   
    commentstyle=\color{codegreen},
    keywordstyle=\color{magenta},
    numberstyle=\tiny\color{codegray},
    stringstyle=\color{codepurple},
    basicstyle=\ttfamily\footnotesize,
    breakatwhitespace=false,         
    breaklines=true,                 
    captionpos=b,                    
    keepspaces=true,                 
    numbers=left,                    
    numbersep=5pt,                  
    showspaces=false,                
    showstringspaces=false,
    showtabs=false,                  
    tabsize=2
}

\lstset{language=Python, style=mystyle}

\usepackage{mathtools}

\usepackage{wasysym}
\usepackage[margin=1.5in]{geometry} 
\usepackage{enumerate}
\index{\usepackage}\usepackage{multicol}

\newcommand{\N}{\mathbf{N}}
\newcommand{\Z}{\mathbb{Z}}

\newcommand{\R}{\mathbf{R}}
\newcommand{\C}{\mathbf{C}}
\newcommand{\Pbb}{\mathbb{P}}
\newcommand{\Fcal}{\mathcal{F}}
\newcommand{\Lcal}{\mathcal{L}}
\newcommand{\Acal}{\mathcal{A}}
\newcommand{\Ecal}{\mathcal{E}}
\newcommand{\Ebb}{\mathbb{E}}
\newcommand{\Qbb}{\mathbb{Q}}


\renewcommand{\mathbf}{\mathbold}

\newenvironment{theorem}[2][Theorem]{\begin{trivlist}
  \item[\hskip \labelsep {\bfseries #1}\hskip \labelsep {\bfseries #2.}]}{\end{trivlist}}
\newenvironment{lemma}[2][Lemma]{\begin{trivlist}
  \item[\hskip \labelsep {\bfseries #1}\hskip \labelsep {\bfseries #2.}]}{\end{trivlist}}
\newenvironment{exercise}[2][Exercise]{\begin{trivlist}
  \item[\hskip \labelsep {\bfseries #1}\hskip \labelsep {\bfseries #2.}]}{\end{trivlist}}
\newenvironment{reflection}[2][Reflection]{\begin{trivlist}
  \item[\hskip \labelsep {\bfseries #1}\hskip \labelsep {\bfseries #2.}]}{\end{trivlist}}
\newenvironment{proposition}[2][Proposition]{\begin{trivlist}
  \item[\hskip \labelsep {\bfseries #1}\hskip \labelsep {\bfseries #2.}]}{\end{trivlist}}
\newenvironment{corollary}[2][Corollary]{\begin{trivlist}
  \item[\hskip \labelsep {\bfseries #1}\hskip \labelsep {\bfseries #2.}]}{\end{trivlist}}

\newenvironment{definition}[2][Definition]{\begin{trivlist}
  \item[\hskip \labelsep {\bfseries #1}\hskip \labelsep {\bfseries #2.}]}{\end{trivlist}}

\definecolor{solar}{rgb}{0.9960, 0.9960, 0.9647}

\begin{document}
  \pagecolor{solar}
	
  \renewcommand{\qedsymbol}{\smiley}
	\title{Investments Class \\ Problem set 9}
	\author{Daniel Grosu, William Martin, Denis Steffen}
		
\maketitle

\begin{exercise}{1}{Mean-variance investing with fixed, linear-proportional, and quadratic trans- action costs}

  To optimize the mean-variance investing for $X_1$, we have two cases because the indicator function is not differentiable. We can define the Lagrangian function as follows:
  $$ \mathcal{L} = \begin{cases}
    X_1\mu - \frac{\gamma}{2}X_1^2\sigma^2 - (b_0 + (X_1-X_0)b_1+\frac{1}{2}\lambda(X_1-X_0)^2), \text{ if } X_1> X_0 \\
    X_1\mu - \frac{\gamma}{2}X_1^2\sigma^2 - (b_0 + (X_0-X_1)b_1+\frac{1}{2}\lambda(X_1-X_0)^2), \text{ if } X_1< X_0
  \end{cases}$$ and so we can differentiate with respect to $X_1$:
  $$ \frac{\partial\mathcal{L}}{\partial X_1} = \begin{cases}
    \mu - \gamma X_1\sigma^2 - (b_1+\lambda(X_1-X_0)), \text{ if } X_1> X_0 \\
    \mu - \gamma X_1\sigma^2 - (- b_1+\lambda(X_1-X_0)), \text{ if } X_1< X_0
  \end{cases}$$

  To find the optimal $X_1$, we equate the two equations to $0$ and find: 
  $$ X_1 = \begin{cases}
    \frac{\mu - b_1 + \lambda X_0}{\gamma\sigma^2+\lambda}, \text{ if } X_1> X_0 \\
    \frac{\mu + b_1 + \lambda X_0}{\gamma\sigma^2+\lambda}, \text{ if } X_1< X_0
  \end{cases}$$

  We know from the lecture course that trading is optimal when $X_0$ does not lie in a no-trade interval that comes from the linear and quadratic transaction costs contraints. Indeed, the linear constraint (alone) gives the no-trade interval $[\frac{\mu- b_1}{\gamma\sigma^2},\frac{\mu+b_1}{\gamma\sigma^2}]$ and the quadratic constraint (alone) gives an optimal value for $X_1 = \frac{\mu+\lambda X_0}{\sigma^2\gamma + \lambda}$. The quadratic constraint does not give any no-trade region as there is a unique solution. So these two constraints give a certain no-trade interval $[\tilde{\underline{X}},\tilde{\overline{X}}]$. 

  In addition, we have a fixed cost constraint and this gives us a second no-trade interval, because we have to make sure that trading is profitable when paying the cost $b_0$ in opposite to not trading. So we have a second interval $[\accentset{\circ}{\underline{X}},\accentset{\circ}{\overline{X}}]$ and then our no-trade region is the domain $ [\tilde{\underline{X}},\tilde{\overline{X}}]\cup [\accentset{\circ}{\underline{X}},\accentset{\circ}{\overline{X}}]$. 

  The fixed cost constraint (only) gives the no-trade interval: $[\frac{\mu}{\sigma^2\gamma} -\sqrt{\frac{2b_0}{\gamma\sigma^2}},\frac{\mu}{\sigma^2\gamma} +\sqrt{\frac{2b_0}{\gamma\sigma^2}}]$. And we can see that the number $\frac{\mu}{\sigma^2\gamma}$ is in the two intervals, so they are not disjoint. 
  
  Thus the no-trade region is an interval $[\underline{X},\overline{X}]$ depending on the parameters, $\mu, b_0, b_1,\gamma, \lambda$ and $\sigma^2$. 

  Now we need to compute the trading speed and the aim portfolio such that: $X_1 = \tau aim + (1-\tau)X_0$

\end{exercise}

\newpage

\begin{exercise}{2}
\end{exercise}

\newpage

\begin{exercise}{3}{Optimal Dynamic trading of a single asset with linear-proportional price impact (quadratic transaction costs)}
\end{exercise}
   
\end{document}



\appendix

