\documentclass[10pt]{article}

\usepackage[T1]{fontenc}
\usepackage[utf8]{inputenc}
%\usepackage{beton}
%\usepackage{ccfonts}
%\usepackage{concrete}
\usepackage{concmath}
\usepackage{eulervm}
\usepackage{amsmath,amsthm,amssymb}
\usepackage{mathtools}
\usepackage{multicol}
\usepackage{marginnote}
\usepackage{pgfplots}
\usepackage{float}
\usepackage{hyperref}
\usepackage{bbm}
\usepackage{booktabs}
\usepackage{xcolor-solarized}
\usepackage{xcolor}
\pgfplotsset{compat=1.5}

\usepackage{listings}
\usepackage{xcolor}
\definecolor{codegreen}{rgb}{0,0.6,0}
\definecolor{codegray}{rgb}{0.5,0.5,0.5}
\definecolor{codepurple}{rgb}{0.58,0,0.82}
\definecolor{backcolour}{rgb}{0.95,0.95,0.92}
\lstdefinestyle{mystyle}{
    backgroundcolor=\color{backcolour},   
    commentstyle=\color{codegreen},
    keywordstyle=\color{magenta},
    numberstyle=\tiny\color{codegray},
    stringstyle=\color{codepurple},
    basicstyle=\ttfamily\footnotesize,
    breakatwhitespace=false,         
    breaklines=true,                 
    captionpos=b,                    
    keepspaces=true,                 
    numbers=left,                    
    numbersep=5pt,                  
    showspaces=false,                
    showstringspaces=false,
    showtabs=false,                  
    tabsize=2
}

\lstset{language=Python, style=mystyle}

\usepackage{mathtools}

\usepackage{wasysym}
\usepackage[margin=1.5in]{geometry} 
\usepackage{enumerate}
\index{\usepackage}\usepackage{multicol}

\newcommand{\N}{\mathbf{N}}
\newcommand{\Z}{\mathbb{Z}}

\newcommand{\R}{\mathbf{R}}
\newcommand{\C}{\mathbf{C}}
\newcommand{\Pbb}{\mathbb{P}}
\newcommand{\Fcal}{\mathcal{F}}
\newcommand{\Lcal}{\mathcal{L}}
\newcommand{\Acal}{\mathcal{A}}
\newcommand{\Ecal}{\mathcal{E}}
\newcommand{\Ebb}{\mathbb{E}}
\newcommand{\Qbb}{\mathbb{Q}}


\renewcommand{\mathbf}{\mathbold}

\newenvironment{theorem}[2][Theorem]{\begin{trivlist}
  \item[\hskip \labelsep {\bfseries #1}\hskip \labelsep {\bfseries #2.}]}{\end{trivlist}}
\newenvironment{lemma}[2][Lemma]{\begin{trivlist}
  \item[\hskip \labelsep {\bfseries #1}\hskip \labelsep {\bfseries #2.}]}{\end{trivlist}}
\newenvironment{exercise}[2][Exercise]{\begin{trivlist}
  \item[\hskip \labelsep {\bfseries #1}\hskip \labelsep {\bfseries #2.}]}{\end{trivlist}}
\newenvironment{reflection}[2][Reflection]{\begin{trivlist}
  \item[\hskip \labelsep {\bfseries #1}\hskip \labelsep {\bfseries #2.}]}{\end{trivlist}}
\newenvironment{proposition}[2][Proposition]{\begin{trivlist}
  \item[\hskip \labelsep {\bfseries #1}\hskip \labelsep {\bfseries #2.}]}{\end{trivlist}}
\newenvironment{corollary}[2][Corollary]{\begin{trivlist}
  \item[\hskip \labelsep {\bfseries #1}\hskip \labelsep {\bfseries #2.}]}{\end{trivlist}}

\newenvironment{definition}[2][Definition]{\begin{trivlist}
  \item[\hskip \labelsep {\bfseries #1}\hskip \labelsep {\bfseries #2.}]}{\end{trivlist}}

\definecolor{solar}{rgb}{0.9960, 0.9960, 0.9647}

\begin{document}
  \pagecolor{solar}
	
  \renewcommand{\qedsymbol}{\smiley}
	\title{Investments Class \\ Problem set 9}
	\author{Daniel Grosu, William Martin, Denis Steffen}
		
\maketitle

\begin{exercise}{1}{Mean-variance investing with fixed, linear-proportional, and quadratic trans- action costs}
\end{exercise}

\newpage

\begin{exercise}{2}
\end{exercise}

a) We solve for the optimal portfolio when there is one single risky asset and the risk-less asset. Thus, maximizing the mean-variance objective function gives us the optimal portfolio (one dimensional) if we decide to long the risky asset $X_1$:

\begin{align*}
	max \quad C(X_1) = R_f + X_1(\mu - R_f) - \frac{\gamma}{2} \Sigma X_1^2 - (X_1 - X_0)b
\end{align*}

Solving for the optimal portfolio $X_1^*$ 

\begin{align*}
	& \mu - R_f - \gamma \Sigma X_1^* - b = 0 \\
	& X_1^* = \frac{1}{\gamma} \Sigma^{-1} (\mu - R_f  - b)
\end{align*}

If it is optimal from the initial dollar position $X_0$ to buy more of $X_1$, then this can be done as long as as $X_0 < X_1^*$. We thus have $X_1^* = X_L$ where we should trade. In other words

\begin{align*}
	X_0 < X_L = \frac{1}{\gamma} \Sigma^{-1} (\mu - R_f  - b)
\end{align*}

Shorting $X_1$ is equivalent to maximizing the following mean-variance objective function:

\begin{align*}
	max \quad C(X_1) = R_f + X_1(\mu - R_f) - \frac{\gamma}{2} \Sigma X_1^2 - (-X_1 + X_0)b
\end{align*}

Computing the first order condition with respect to $X_1$ gives us:

\begin{align*}
	X_1^* =\frac{1}{\gamma} \Sigma^{-1} (\mu - R_f + b)
\end{align*}

Similarly, we can sell the risky asset $X_1$ from initial dollar position $X_0$ as long as $X_0 > X_1^*$, which gives us $X_h$:

\begin{align*}
	X_0 > X_h = \frac{1}{\gamma} \Sigma^{-1} (\mu - R_f + b)
\end{align*}

The no-trade region is thus the inverse of the optimal trading region, that is $[X_l, X_h]$, where we should not trade.

\bigbreak

b) With two assets to trade, we can draw all trade possibilities in a table:

\begin{center}
\begin{tabular}{|l|c|c|c|}
\hline
           & \multicolumn{1}{l|}{buy 1} & \multicolumn{1}{l|}{no trade 1} & \multicolumn{1}{l|}{sell 1} \\ \hline
sell 2      & bs  & ns & ss \\ \hline
no trade 2 & bn  & nn & sn \\ \hline
buy 2     & bb & nb & sb  \\ \hline
\end{tabular}
\end{center}

The no-trade region is thus, similarly to a), the inverse region of the do-trade region defined by the sets $\{bb, bn, bs, nb, ns, sb, sn, ss\}$. We thus need to define these 8 regions first in order to define the no-trade region.

\smallbreak

We need to control the sign of each component of $(X_1 - X_0)^T$. To this end we introduce the diagonal matrix \textbf{A}:

\begin{align*}
	A = \begin{pmatrix}
a_{1} & 0\\
0 & a_{2}
\end{pmatrix}
\end{align*}

Where components $a_{11}$ and $a_{22}$ can either take values -1, 1 or 0 to emulate either a sell, buy or no-trade position of each risky asset. We can therefore rewrite the objective function as follows

\begin{align*}
	max \quad C(X_1) = R_f  + X_1^T(\mu - R_f)  - \frac{\gamma}{2} X_1^T \Sigma X_1  - (X_1  - X_0)^TAb
\end{align*}

Taking the first case of the table, that is buy risky asset 1 and 2 ($bb$) and maximizing the modified mean-variance objective function, taking the first order condition: 

\begin{align*}
	& \mu - R_f - \gamma \Sigma X_1 - Ab = 0\\
	& X_{bb} = X_1^*(A) = \frac{1}{\gamma}\Sigma^{-1} (\mu - R_f - Ab)
\end{align*}

Where 

\begin{align*}
	A = \begin{pmatrix}
	1 & 0\\	
	0 & 1
\end{pmatrix}
\end{align*}

Let us denote by $X_0^1$ and $X_0^2$ the initial dollar position of the first and second risky assets respectively and similarly $X_1^1$ and $X_1^2$ their final dollar position. The optimal trading region for the $bb$ case is thus when $bb = \{X_0^1 < X_{bb}^1 \cap  X_0^2 < X_{bb}^2\}$.

\bigbreak

The other 3 corners of the table can be expressed by following the same procedure. For the $sb$ case. We have 

\begin{align*}
	X_{sb} &= X_1^*(A) \\
	A &=  \begin{pmatrix}
	-1 & 0\\	
	0 & 1
\end{pmatrix}
\end{align*}

Therefore $sb = \{X_0^1 > X_{sb}^1 \cap  X_0^2 < X_{sb}^2\}$. For the $bs$ case we have 

\begin{align*}
		X_{bs} &= X_1^*(A) \\
	A &=  \begin{pmatrix}
	1 & 0\\	
	0 & -1
\end{pmatrix}
\end{align*}

Therefore $bs = \{X_0^1 < X_{bs}^1 \cap  X_0^2 > X_{bs}^2\}$. Finally for the $ss$ case we have 

\begin{align*}
		X_{ss} &= X_1^*(A) \\
	A &=  \begin{pmatrix}
	-1 & 0\\	
	0 & -1
\end{pmatrix}
\end{align*}

Therefore $ss = \{X_0^1 > X_{ss}^1 \cap  X_0^2 > X_{ss}^2\}$.

\bigbreak

We now need to defined the trading regions where only one risky asset is traded. For the $bn$ case, we leave the second risky asset untouched, that is $X_1^2 = X_0^2$: its final and initial dollar positions are identical. The modified mean-variance objective function to maximize becomes the following ($a_1 = 1$ and $a_2 = 0$ in the matrix $A$)

\begin{align*}
	max \quad C(X_1^1) &= R_f + X_1^T(\mu - R_f) - \frac{\gamma}{2}X_1^T\Sigma X_1 - (X_1 - X_0)^T \begin{pmatrix}
	1 & 0\\	
	0 & 0
\end{pmatrix} b \\
&= R_f + X_1^1(\mu_1 - R_f) + X_0^2(\mu_2 - R_f) - \frac{\gamma}{2}  \begin{pmatrix}
	X_1^1 & X_0^2
\end{pmatrix} \begin{pmatrix}
	\sigma_1^2 & \rho \sigma_1 \sigma_2\\	
	\rho \sigma_1 \sigma_2 & \sigma_2^2
\end{pmatrix} \begin{pmatrix}
	X_1^1 \\	
	X_0^2 
\end{pmatrix} \\
& - \begin{pmatrix}
	X_1^1 - X_0^1 & X_0^2 - X_0^2
\end{pmatrix} \begin{pmatrix}
	b_1 \\	
	b_2 
\end{pmatrix}\\
&= R_f + X_1^1(\mu_1 - Rf - b_1) + X_0^2(\mu_2 - R_f) - \frac{\gamma}{2}\left( X_1^1 X_1^1 \sigma_1^2 + X_0^2 X_0^2 \sigma_2^2 + 2 X_1^1X_0^2\rho \sigma_1 \sigma_2 \right)
\end{align*}

Taking the first order condition with respect to $X_1^1$ gives us

\begin{align*}
	& \mu_1 - R_f - b_1 - \gamma X_1^1\sigma_1^2 - \gamma X_0^2 \rho \sigma_1 \sigma_2 = 0	\\
	& X_{bn}^1(X_0^2) =  X_1^{1*}(X_0^2)  = \frac{\mu_1 - R_f - b_1}{\gamma \sigma_1^2} - \frac{X_0^2 \rho  \sigma_2}{\sigma_1}
\end{align*}

Moreover, since we are not trading risky asset 2, this must mean that for the  $bn$ case it must be in the no-trade region knowing that risky asset 1 is in a long position: $X_{bb}^2 \leq X_0^2 \leq X_{bs}^2$ (result similar to the first part $a$). The  set $bn $ is thus the intersection between not trading risky asset 2 (conditional on longing risky asset 1) and buy risky asset 1 (conditional on not trading risky asset 2) until reaching $X_{bn}^1(X_0^2)$. Therefore, $bn = \{ X_{bb}^2 \leq X_0^2 \leq X_{bs}^2 \cap X_0^1 < X_{bn}^1(X_0^2) \}$.

\bigbreak

We can derive similar results for the other 3 quadrants of the table, where one risky asset is traded and the other risky asset is not traded. For $sn$, we optimally sell risky asset 1 and don't trade risky asset 2

\begin{align*}
	X_{sn}^1(X_0^2) =  X_1^{1*}(X_0^2)  = \frac{\mu_1 - R_f + b_1}{\gamma \sigma_1^2} - \frac{X_0^2 \rho  \sigma_2}{\sigma_1}
\end{align*}

Following the same argument than for $bn$, we obtain the set $sn = \{ X_{sb}^2 \leq X_0^2 \leq X_{ss}^2 \cap X_0^1 > X_{sn}^1(X_0^2) \}$. For $nb$, we optimally buy risky asset 2 while not trading risky asset 1

\begin{align*}
	X_{nb}^2(X_0^1) =  X_1^{2*}(X_0^1)  = \frac{\mu_2 - R_f - b_2}{\gamma \sigma_2^2} - \frac{X_0^1 \rho  \sigma_1}{\sigma_2}
\end{align*}

The set is thus $nb = \{ X_{bb}^1 \leq X_0^1 \leq X_{sb}^1 \cap X_0^2 < X_{nb}^2(X_0^1) \}$.

For $ns$, we optimally sell risky asset 2 while not trading risky asset 1

\begin{align*}
	X_{ns}^2(X_0^1) =  X_1^{2*}(X_0^1)  = \frac{\mu_2 - R_f + b_2}{\gamma \sigma_2^2} - \frac{X_0^1 \rho  \sigma_1}{\sigma_2}
\end{align*}

The set is thus $ns = \{ X_{bs}^1 \leq X_0^1 \leq X_{ss}^1 \cap X_0^2 > X_{ns}^2(X_0^1) \}$.

\bigbreak

Finally the no-trade region is the inverse of all trading regions. That is 

\begin{align*}
	nn = {\overline{bb \cup sb \cup bs \cup ss \cup bn \cup sn \cup nb \cup ns}}
\end{align*}

\newpage

\begin{exercise}{3}{Optimal Dynamic trading of a single asset with linear-proportional price impact (quadratic transaction costs)}
\end{exercise}
   
\end{document}



\appendix


